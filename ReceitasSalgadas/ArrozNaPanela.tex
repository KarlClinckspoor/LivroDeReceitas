\receitaemoji[\testado \karlAprova \karenAprova]{Arroz na panela}{

	\begin{itemize}
		\item Arroz normal. 1 medida de 3/4 de copo é o suficiente para 1 pessoa.
		\item 0,5 cebola grande
		\item 1 ponta de faca de alho picado ($\pm$ 1 dente)
		\item 2 viradas completas do moedor de sal, talvez 3
	\end{itemize}
}{
	\begin{enumerate}

		\item Lavar o arroz na peneira bastante para tirar muito do pó branco.
		      Espalhar o arroz para ajudar a secar. Fazer isso com antecedência para que o
		      arroz esteja relativamente seco antes de colocar na panela.
		\item Cortar a cebola em pedaços pequenos, manualmente ou no processador
		\item Aquecer a panela e colocar o óleo, um fio. Fogo alto.
		\item Fritar a cebola até ela ficar um tanto douradinha.
		\item Colocar o alho até soltar o aroma
		\item Colocar o sal
		\item Colocar o arroz e fritar um pouquinho.
		\item Cobrir o arroz com 1 falange de água filtrada\footnote{Inspirado no
			      método do arroz japonês}
		\item Esperar o arroz ferver em fogo alto e o nível da água atingir o nível do
		      arroz.
		\item Abaixar o fogo para o mínimo e cobrir com uma tampa.
		\item Em torno de 10-12 minutos o arroz deve estar pronto. Resistir remover a
		      tampa.
	\end{enumerate}
}
