\section{Tabelas}
\begin{table}[h]
	\centering
	\begin{tabular}[h]{c c}
		\hline
		Carne                  & Temperatura              \\
		\hline
		Vaca                   & 145 \grau F = 63 \grau C \\
		Porco                  & 145 \grau F = 63 \grau C \\
		Peixes e frutos do mar & 145 \grau F = 63 \grau C \\
		Carnes moídas          & 160 \grau F = 71 \grau C \\
		Ovos                   & 160 \grau F = 71 \grau C \\
		Frango                 & 165 \grau F = 74 \grau C \\
		Sobras                 & 165 \grau F = 74 \grau C \\
		\hline
	\end{tabular}
	\caption{Temperaturas mínimas do interior de carnes para checar se estão cozidas.
		Geralmente devem ficar por ao menos 3 minutos nessa temperatura}
	\label{tab:temperaturas_cozimento}
\end{table}

\begin{table}[h]
	\centering
	\begin{tabular}[h]{c c}
		\hline
		1 colher de chá  & $\approx$ 5 mL                    \\
		1 colher de sopa & $\approx$ 15 mL                   \\
		1 onça           & $\approx$ 28 g                    \\
		1 copo           & $\approx$ 240 mL                  \\
		1 kg             & $\approx$ 2.20 libra (lb)         \\
		1 libra (lb)     & 0.45 kg                           \\
		1 \emph{quart}   & 1.05 L                            \\
		\hline
		1 \grau C        & $\frac{5}{9} \left( F-32 \right)$ \\
		1 \grau F        & $\left(\frac{9}{5} C\right)+32$   \\
		\hline
		225 \grau F      & 105 \grau C                       \\
		250 \grau F      & 120 \grau C                       \\
		275 \grau F      & 135 \grau C                       \\
		300 \grau F      & 150 \grau C                       \\
		325 \grau F      & 165 \grau C                       \\
		350 \grau F      & 180 \grau C                       \\
		375 \grau F      & 190 \grau C                       \\
		400 \grau F      & 200 \grau C                       \\
		425 \grau F      & 220 \grau C                       \\
		450 \grau F      & 230 \grau C                       \\
		475 \grau F      & 245 \grau C                       \\
		\hline
	\end{tabular}
	\caption{Conversão de unidades}
	\label{tab:conversao_unidades}
\end{table}

\begin{table}[h]
	\centering
	\begin{tabular}[h]{c c}
		\hline
		tablespoon    & colher de sopa       \\
		teaspoon      & colher de chá        \\
		buttermilk    & leitelho             \\
		cup           & xícara               \\
		baking powder & fermento em pó       \\
		baking soda   & bicarbonato de sódio \\
		\hline
	\end{tabular}
	\caption{Traduções de frases comuns}
	\label{tab:traducoes_comuns}
\end{table}

% \rotatebox{90}{
\begin{landscape}
  \begin{table}[h]
    \centering
    \begin{tabular}[h]{c c c c c c}
      \hline
      Comida & Quantidade & Nível água & Tempo fervura & Tempo cozimento & Obs \\
      \hline
      Feijão & 2 tigelas & Metade & 16 min & $<$ 20 min & Deixar uns 10-15 min hidratando\\
      Beterraba & 4 beterrabas grandes & Metade & $\sim$ 15 min & $\sim$ 12 min & - \\
      Acém & 900 g & Metade & - & $<$ 30 min & 30 min foi demais \\
      \hline
    \end{tabular}
    \caption{Tempos de cozimento na panela de pressão}
    \label{tab:tempos_cozimento_pp}
  \end{table}
\end{landscape}
% }
\clearpage

\section{Receitas rápidas}

\begin{itemize}
	\item Linguiça calabresa na frigideira fica boa por 5-10 minutos no fogo médio com tampa, virando
	      constantemente para não queimar. Se estiver bem tostada por fora mas fria por dentro, abaixar bem o fogo e
	      tampar. Checar a temperatura no interior com o termômetro.
	\item Para cozinhar beterrabas sem panela de pressão, cortar as mesmas pela metade, se estiverem muito
	      grandes, cobrir com água e ligar o fogo alto. Colocar um cronômetro a cada 30 minutos para alertar e ir
	      checar o nível da água e a textura, pois não pode tampar. No teste feito, 1h20 foi suficiente para
	      cozinhá-las ao ponto de poder passar uma faca por uma facilmente.
	\item Para cozinhar batata doce, colocar a batata cortada na água fria ainda. Demora cerca de 30 minutos
	      entre o tempo que a água começa a ferver e a batata estar cozida.
	\item Para cozinhar mandioca, cortar a mandioca para fazer caber, e não ter pedaços grandes demais.
	      Colocar na água e esquentar no máximo. Demora cerca de 1h para cozinhar. Depois, se quiser fritar no
	      airfryer, recobrir com óleo e sal e fritar a 200 \grau C por 10-14 minutos.
	\item Buttermilk, ou leitelho, é um derivado de leite ácido com pouca gordura. Para fazer um buttermilk
	      falso, pode-se utilizar leite e suco de limão/vinagre. Misturar e deixar por 10 minutos na bancada.
	\item Para fazer ghee, colocar a manteiga numa panela mais alta, e colocar o fogo o mais baixo possível. No
	      caso aqui, isso ainda é alto demais, então achar uma maneira de levantar a panela. Deixar evaporando por
	      um tempo, depois coar por um papel filtro. Tomar cuidado com o choque térmico da manteiga quente com o
	      vidro, para não estilhaçar.
	\item Para fazer grão de bico cozido, pernoitar na água, depois cozinhar com água, sal e louro por 40
	      minutos. O líquido que resta poderia, em teoria, ser batido para fazer algo similar a chantilly, mas
	      não é tão trivial. Eu testei, mas provavelmente precisaria de uma máquina, e ter colocado menos água.
\end{itemize}
\clearpage

\section{Informações, observações aleatórias}

\begin{itemize}
	\item Após utilizar feijão para fazer peso em alguma massa, não pode mais ser usado para cozinhar.
	\item Pimentão e cenoura, se ficarem fora de sacos na geladeira, ficam moles e enrugados. Cebola fica com
	      cheiro de geladeira e seca, esquisita, e a geladeira fica fedida. Alface fica super murcha, e tentar fazer
	      osmose deixando imersa em água não adianta muito.
\end{itemize}

\subsection*{Creme de leite e chantilly}
\begin{itemize}
	\item O creme de leite pode ter uma concentração de gordura na parte de cima. Então ele fica mais
	      consistente em cima do que em baixo. Um pouco de acidez é normal.
	\item O creme de leite está estragado quando fica amarelado
	\item Para fazer chantilly, o creme deve estar fresquíssimo
	\item O creme pode ser guardado na freezer por longos períodos. Tirar do frasco, homogeneizar, separar em
	      copos de $\pm$ 100 mL e guardar no freezer. Sempre deixar espaço para a expansão do gelo.
\end{itemize}

\subsection*{Bater clara em neve}
\begin{itemize}
	\item Para bater clara em neve sem equipamento, o processo parece mais penoso do que realmente é, se possuir
	      um Fouet. Colocar as claras numa tigela grande, e fazer movimentos com o braço e o punho também. Demorou
	      pouco da vez que eu fiz, 5 minutos, mais ou menos.
\end{itemize}




\subsection*{Tempos e métodos de requentamento}

\subsubsection*{Pizza de massa tradicional}
\begin{itemize}
	\item 160 \grau C por 8 minutos resultou numa pizza meio mole
	\item 200 \grau C por 10 minutos queimou um pouco, ficou pior que a 160 \grau C.
	\item 180 \grau C por 8 minutos parece que ficou bom
	\item No microondas ficou horrível.
\end{itemize}

\subsubsection*{Pan pizza}
\begin{itemize}
	\item De forma geral, 180 \grau C por 10 minutos serve. Microondas fica pior, mas comível.
	\item As pizzas do Pizza Hut e Domino's requentam melhor que as pizzas tradicionais.
\end{itemize}

\subsubsection*{Esfiha fechada}
\begin{itemize}
	\item 180 \grau C por 5-8 minutos ficou bom.
\end{itemize}

%%% Local Variables:
%%% mode: latex
%%% TeX-master: "main"
%%% End: