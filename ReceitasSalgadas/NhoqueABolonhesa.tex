\receitaemoji[\testado \karlAprova \karenAprova]{Nhoque a bolonhesa}{

	\textsc{Nhoque}

	\begin{itemize}
		\item 600g de batatas Asterix
		\item 1 ovo
		\item 2 xícaras de chá de parmesão ralado fino
		\item 1 colher de sopa de manteiga
		\item Pimenta síria a gosto
		\item 1 colher de sopa de farinha de trigo
		\item 1 fio de azeite e sal para a água do cozimento
	\end{itemize}

	\textsc{Molho bolonhesa} \label{rec:molho_bolonhesa}

	\begin{itemize}
		\item 1 colher de sopa de óleo ou azeite
		\item 1 cebola média picada
		\item 2 dentes de alho
		\item 1 kg de carne moída
		\item Orégano
		\item Pimenta do reino
		\item Colorau
		\item Sal
		\item 1 lata de extrato de tomate
		\item 1 sachê de molho de tomate
		\item 1 xícara de chá de água
	\end{itemize} }{

	\textsc{Nhoque}

	\begin{enumerate}
		\item Cozinhar as batatas, amassar ainda quente. (com casca?)
		\item Aguardar as batatas esfriarem. Adicionar os outros ingredientes até
		      formar uma massa tipo pão leve. Se necessário, adicionar mais farinha (o que
		      observar?)
		\item Ferver água com um fio de azeite.
		\item Fazer rolinhos e cortar a massa em tubos curtos. Colocar na água
		      fervente.
		\item Quando a massa boiar, retirar da água.
	\end{enumerate}

	\textsc{Molho bolonhesa}
	\begin{enumerate}
		\item Numa panela média, refogue a cebola e o alho no óleo ou no azeite até
		      dourar.
		\item Em seguida adicione a carne moída e tempere com orégano, pimenta do
		      reino, colorau, sal a gosto.
		\item Mexer para misturar bem e refogar por aproximadamente 10 minutos.
		\item Adicionar o extrato, molho e água.
		\item Misturar, tampar a panela e deixar cozinhar por mais 5 minutos.
		\item Desligar o fogo
	\end{enumerate}

	\fotoreceita{0.7\textwidth}{./Fotos/Nhoque} }
