\receitaemoji[\testado \karlAprova \karenAprova]{Conchiglioni de queijo}{
	\begin{itemize}
		\item Aproximadamente 50 Conchiglioni
		\item 1 cilindro de ricota fresca. Mais molinha é melhor (cerca de 200g?)
		\item 1 pote de creme de ricota.
		\item 100g de mussarella ralada
		\item 100g de provolone ralado
		\item 25g de parmesão ralado para recheio, um pouco para polvilhar em cima
		\item molho de tomate pronto. 1 saquinho para pouco molho, 2 para banhar bem.
		\item sal
		\item nozes picadas, aproximadamente 8
	\end{itemize} }{
	\begin{enumerate}
		\item Cozinhar os conchiglioni em água com sal até ficarem no ponto.
		\item Enquanto cozinha, misturar todos os queijos juntos, com as nozes. Provar
		      se tem sal suficiente e colocar um queijo ou outro para atingir o sabor.
		      Senão, colocar sal, mas por experiência, para meu gosto, isso não é
		      necessário.
		\item Ao terminar, coar o macarrão, deixar esfriar um pouco. Começar a aquecer
		      o forno a 200 $^\circ$C.
		\item Numa travessa grande, cobrir a base com molho de tomate, para evitar do
		      macarrão queimar durante o assamento.
		\item Preencher generosamente os conchiglioni com o recheio. Colocar na
		      travessa da maneira mais eficiente possível, porque 50 conchiglioni é
		      bastante.
		\item Jogar molho por cima deles, tentando ser homogêneo, mas não é
		      necessário. Após, salpicar parmesão a gosto.
		\item Colocar no forno por 15 minutos. Se tiver duas travessas, retirar a
		      travessa de baixo, mais perto do fogo, antes, para que não queime. O
		      interessante é derreter o recheio e aquecer tudo, pois a comida já está
		      tecnicamente pronta para ser comida.
		\item Servir e comer imediatamente. O que sobrar pode ser resfriado ou
		      congelado e posteriormente aquecido no micro ondas que não há perda de
		      sabor.
	\end{enumerate}

	\fotoreceita{0.5\textwidth}{./Fotos/conchiglioni} }
