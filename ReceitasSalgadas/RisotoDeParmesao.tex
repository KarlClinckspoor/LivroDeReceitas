\receitaemoji[\testado \karlAprova \karenAprova]{Risoto de parmesão}{
	\begin{itemize}
		\item 2,5 copos de caldo de frango (vide receita \ref{rec:caldo_frango})
		\item 0,75 copos de água
		\item 2 colheres de sopa de manteiga
		\item 0,5 cebola grande, finamente cortada
		\item Sal, pimenta
		\item 0,5 dente de alho, cortado finamente
		\item 1 copo de arroz para risoto (arbório ou outro)
		\item 25-30g de parmesão ralado
		\item 1 colher de sopa de salsinha e cebolinha
		\item 0,5 colher de sopa de suco de limão
	\end{itemize} }{
	\begin{enumerate}
		\item Ferver o caldo e a água em panela grande em fogo quente. Depois reduzir
		      o calor para um fervor gentil.
		\item Derreter metade da manteiga numa panela\footnote{De preferência de
			      ferro, mas pode ser alguma com uma massa térmica maior que uma panela
			      comum} em calor médio. Colocar a cebola e 0,75 colher de sopa de sal. Ir
		      mexendo por 5 a 7 minutos.
		\item Colocar o alho até ficar cheiroso, 30 segundos.
		\item Colocar o arroz e cozinhar, mexendo lentamente até os grãos ficarem
		      translúcidos nas bordas, 3 minutos.
		      \begin{itemize}
			      \item Cuidado com deixar o arroz queimar nesta etapa! Mexer bastante.
		      \end{itemize}
		\item Colocar o caldo aquecido até cobrir o arroz. Reduzir o calor para
		      médio-baixo. Ficar mexendo até absorver. Quando secar, adicionar mais caldo
		      em parcelas bem pequenas. Sempre ficar mexendo para garantir que não vai
		      queimar. Continuar até o arroz ficar cremoso e cozido, ou acabar o caldo.
		      16-19 minutos.
		\item Colocar o parmesão e mexer. Remover do calor, cobrir, e deixar parado
		      por 5 minutos.
		\item Colocar o resto da manteiga, salsinha, cebolinha e limão.
		      \begin{itemize}
			      \item O limão não foi muito bem quisto pela Karen. Dá para colocar a
			            salsinha e cebolinha, separar em duas metades, e em uma delas colocar 0,25
			            colher de sopa do suco de limão.
		      \end{itemize}
	\end{enumerate}

	\fotoreceita{0.7\textwidth}{./Fotos/Risoto.png} }
