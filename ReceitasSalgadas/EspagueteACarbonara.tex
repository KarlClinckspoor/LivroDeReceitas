\receitaemoji[\testado \karlAprova]{Espaguete a carbonara}{
	\begin{itemize}
		\item 80-90 g de macarrão espaguete n\grau 3 (outros tamanhos são ok)
		\item $\pm$ 3 tiras de bacon
		\item $\pm$ 20 g de Pecorino ou Parmesão
		\item 1 ovo inteiro
		\item Pimenta do reino
	\end{itemize}
}{
	\begin{enumerate}
		\item Colocar a água para esquentar e colocar uma pitada de sal.
		\item Picar o bacon
		\item Numa tigela, bater o ovo com o queijo com um garfo.
		\item Quando a água estiver fervendo, colocar o macarrão.
		\item Após $\pm$ 1-2 minuto, começar a fritar o bacon numa frigideira.
		\item Após os pedaços de bacon ficarem crocantes, remover o excesso da gordura
		\item Checar se o macarrão já está no ponto. Se sim, transferir diretamente da
		      panela para o bacon, com o fogo desligado.
		\item Pegar uma xícara de água fervente na mão esquerda e um fouet na mão
		      direita. Bater o ovo e o queijo a medida que se coloca um fio contínuo de
		      água. Tem que ser lentamente.
		\item Depois de ficar cremoso, colocar no macarrão.
		\item A maneira tradicional de se colocar o ovo e queijo no macarrão com bacon
		      e mexer é difícil de fazer funcionar. Se achar que o ovo está cozido de
		      menos e ligar o fogo, vai coagular. Por essa maneira, eu desenvolvi um método alternativo que,
          possivelmente, faria um italiano ficar enojado. É o seguinte.
          \begin{enumerate}
          \item Fritar o bacon, misturar ele frio no ovo e queijo
          \item Cozinhar o macarrão da maneira normal
          \item Quando estiver perto de terminar, colocar 1/2 xícara da água do macarrão junto com a mistura
            de ovo+queijo+bacon. Mexer com o fouet. Colocar no microondas de 10 em 10 segundos até ele ficar
            cremoso. Temperatura tem que ser $\sim$ 160 \grau F.
          \end{enumerate}

		\item Servir imediatamente.
	\end{enumerate}
}
