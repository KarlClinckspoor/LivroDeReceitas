\section{Tabelas}
\begin{table}[h]
	\centering
	\begin{tabular}[h]{c c}
		\hline
		Carne                  & Temperatura              \\
		\hline
		Vaca                   & 145 \grau F = 63 \grau C \\
		Porco                  & 145 \grau F = 63 \grau C \\
		Peixes e frutos do mar & 145 \grau F = 63 \grau C \\
		Carnes moídas          & 160 \grau F = 71 \grau C \\
		Ovos                   & 160 \grau F = 71 \grau C \\
		Frango                 & 165 \grau F = 74 \grau C \\
		Sobras                 & 165 \grau F = 74 \grau C \\
		\hline
	\end{tabular}
	\caption{Temperaturas mínimas do interior de carnes para checar se estão cozidas.
		Geralmente devem ficar por ao menos 3 minutos nessa temperatura}
	\label{tab:temperaturas_cozimento}
\end{table}

\begin{table}[h]
	\centering
	\begin{tabular}[h]{c c}
		\hline
		1 colher de chá  & $\approx$ 5 mL                    \\
		1 colher de sopa & $\approx$ 15 mL                   \\
		1 onça           & $\approx$ 28 g                    \\
		1 copo           & $\approx$ 240 mL                  \\
		\hline
		1 \grau C        & $\frac{5}{9} \left( F-32 \right)$ \\
		1 \grau F        & $\left(\frac{9}{5} C\right)+32$   \\
		\hline
		225 \grau F      & 105 \grau C                       \\
		250 \grau F      & 120 \grau C                       \\
		275 \grau F      & 135 \grau C                       \\
		300 \grau F      & 150 \grau C                       \\
		325 \grau F      & 165 \grau C                       \\
		350 \grau F      & 180 \grau C                       \\
		375 \grau F      & 190 \grau C                       \\
		400 \grau F      & 200 \grau C                       \\
		425 \grau F      & 220 \grau C                       \\
		450 \grau F      & 230 \grau C                       \\
		475 \grau F      & 245 \grau C                       \\
		\hline
	\end{tabular}
	\caption{Conversão de unidades}
	\label{tab:conversao_unidades}
\end{table}

\clearpage
\section{Informações, observações aleatórias}

\begin{itemize}
	\item Creme de leite e chantilly
	      \begin{itemize}
		      \item O creme de leite pode ter uma concentração de gordura na parte de cima.
		            Então ele fica mais consistente em cima do que em baixo. Um pouco de acidez
		            é normal.
		      \item O creme de leite está estragado quando fica amarelado
		      \item Para fazer chantilly, o creme deve estar fresquíssimo
		      \item O creme pode ser guardado na freezer por longos períodos. Tirar do
		            frasco, homogeneizar, separar em copos de $\pm$ 100 mL e guardar no freezer.
		            Sempre deixar espaço para a expansão do gelo.
	      \end{itemize}

	\item Após utilizar feijão para fazer peso em alguma massa, não pode mais ser
	      usado para cozinhar.
  \item Métodos para requentar pizza do Kingdom (massa tradicional)
    \begin{itemize}
    \item 160 \grau C por 8 minutos resultou numa pizza meio mole
    \item 200 \grau C por 10 minutos queimou um pouco, ficou pior que a 160
      \grau C.
    \item No microondas ficou horrível.
    \end{itemize}
  \item As pizzas do Pizza Hut e Domino's requentam melhor.
  \item Linguiça calabresa na frigideira fica boa por 5-10 minutos no fogo médio
    com tampa, virando constantemente para não queimar. Se estiver bem tostada
    por fora mas fria por dentro, abaixar bem o fogo e tampar. Checar a
    temperatura no interior com o termômetro.
  \item Para cozinhar beterrabas sem panela de pressão, cortar as mesmas pela
    metade, se estiverem muito grandes. Cobrir com água e ligar o fogo alto.
    Colocar um cronômetro a cada 30 minutos para alertar e ir checar o nível da
    água e a textura. No teste feito, 1h20 foi suficiente para cozinhá-las ao
    ponto de poder passar uma faca por uma facilmente.
\end{itemize}

%%% Local Variables:
%%% mode: latex
%%% TeX-master: "main"
%%% End: