\receitaemoji[\karlAprova \karenAprova]{Feijão na panela de pressão}{
  \begin{itemize}
  \item Feijão
  \item Cebola
  \item Alho
  \item Folha de louro
  \item Feijão carioca ou preto
  \item (opcional) linguiça ou bacon
  \end{itemize}
}{
  \begin{itemize}
  \item Deixar 2 tigelas de feijão de molho por 10 minutos, aproximadamente.
  \item Colocar o feijão na panela e completar até o nível mínimo com água. Colocar a folha de louro e a
    linguiça/bacon.
  \item Colocar no fogo máximo até começar a apitar, depois colocar fogo médio
  \item Depois de passar uns 15-17 minutos apitando, remover do fogo, colocar na pia, e levantar o peso com
    um garfo, cuidadosamente, para não se queimar com a pressão. Pode ser que borbulhe e suje tudo em volta.
    Se tiver tempo, deixe resfriar sozinho.
  \item Em uma panela, refogar 1/2 cebola média e 1/2 dente de alho bem picado, junto com um pouco de sal.
  \item Colocar o feijão na panela, amassar um pouco para ajudar no caldo, colocar sal a gosto. Servir quente.
  \end{itemize}
}

\receitaemoji[\testado \karlAprova \karenAprova]{Feijão na panela normal}{
  \begin{itemize}
  \item Feijão
  \item Cebola
  \item Alho
  \item Folha de louro
  \item Feijão carioca ou preto
  \item (opcional) linguiça ou bacon
  \end{itemize}
}{
  \begin{itemize}
  \item Deixar 1 tigela de feijão de molho durante a noite
  \item Fritar 1/2 cebola e 1 dente de alho em bacon ou óleo. Colocar o feijão e folha de louro.
  \item Cobrir com água e provar o sal.
  \item Cozinhar por 40-60 minutos, checando de 15 em 15 depois de 30 min.
  \item Para fazer o caldo, amassar um pouco de feijão e reduzir bem o líquido. É difícil de cozinhar demais
    por esse método.
  \item Colocar o feijão na panela e completar até o nível mínimo com água. Colocar a folha de louro e a
    linguiça/bacon.
  \item Colocar no fogo máximo até começar a apitar, depois colocar fogo médio
  \item Depois de passar uns 15-17 minutos apitando, remover do fogo, colocar na pia, e levantar o peso com
    um garfo, cuidadosamente, para não se queimar com a pressão. Pode ser que borbulhe e suje tudo em volta.
    Se tiver tempo, deixe resfriar sozinho.
  \item Em uma panela, refogar 1/2 cebola média e 1/2 dente de alho bem picado, junto com um pouco de sal.
  \item Colocar o feijão na panela, amassar um pouco para ajudar no caldo, colocar sal a gosto. Servir quente.
  \end{itemize}
}