\receitaemoji[\testado \karlAprova \karenAprova]{Onigiri}{
	\begin{itemize}
		\item Arroz japonês (vide receita \ref{rec:arroz_japones})
		\item Sal
		\item (opcional) Folha de alga
	\end{itemize}

	\textsc{Temperos}
	\begin{itemize}
		\item Óleo de gergelim tostado
		\item Wasabi
		\item Shoyu
		\item Temperinho de arroz japonês
	\end{itemize}

	\textsc{Recheios}
	\begin{itemize}
		\item Bacon
		\item Ovo
	\end{itemize}
}{

	\textsc{Preparo do onigiri}
	\begin{enumerate}
		\item Deixar o arroz esfriar um pouco\footnote{Inspirado no vídeo
			      \texttt{https://www.youtube.com/watch?v=phZTQHEzU6w}}
		\item Molhar a mão na água de torneira e passar um pouco de sal grosso.
		      Evitar colocar muito.
		      \begin{itemize}
			      \item Uma opção interessante é fazer a água já salgada, para diminuir a
			            presença de grãos de sal grosso no onigiri pronto. O resultado
			            é bem menos salgado do que usar sal grosso.
		      \end{itemize}
		\item Pegar um punhado de arroz e colocar na mão, deixar plano.
		\item Fazer uma pequena depressão com os dedos e colocar o recheio.
		\item Fechar as mãos no formato do bolinho. Não pressionar de modo a não
		      aglomerar o arroz.
		\item Servir. Usar algum dos temperos para dar mais sabor, se desejado
	\end{enumerate}

	\textsc{Preparo dos recheios}
	\begin{enumerate}
		\item[Bacon] Cortar o bacon em tiras finas ou cubos pequenos. Fritar somente
		      na própria gordura. Reservar.
		      %  \item[Ovo]
		\item[Ovo] Bater um ovo até ficar bem homogêneo. Esquentar a frigideira até
		      ficar quente. Se puder usar a gordura do bacon, ajuda no sabor. Colocar o
		      ovo e espalhar bem. Continuamente mexer na panela para garantir que a
		      camada de ovo ficará com a mesma espessura. Depois de um lado ficar
		      pronto, virar, cozinhar o outro. Transferir para uma tábua e cortar em
		      tiras ou em quadrados.
		      \begin{itemize}
			      \item Para fazer os pegadores, cortar as bordas redondas para formar
			            um quadrado de ovo.
			      \item Considerar o tamanho do onigiri a ser feito para cortar. O
			            pegador deve passar pela parte de baixo e ir para os dois lados.
			      \item Nos onigiris feitos aqui, tiras com cerca de 3/4 de tamanho
			            servem para isso. O 1/4 restante pode virar recheio, assim como as
			            rebarbas.
		      \end{itemize}
	\end{enumerate}

	% \fotoreceita{0.5\textwidth}{./Fotos/onigiri.jpg}
	% \fotoreceita{0.5\textwidth}{./Fotos/onigiri2.jpg}
	\fotoreceita{0.5\textwidth}{./Fotos/onigiri3.jpg}
}
