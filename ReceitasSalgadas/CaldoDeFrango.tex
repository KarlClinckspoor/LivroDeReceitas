\receitaemoji[\testado \karlAprova \karenAprova]{Caldo de frango\label{rec:caldo_frango}}{
	\begin{itemize}
		\item 3 dentes de alho
		\item 1 cebola pequena
		\item 2 coxas de frango, 2 sobrecoxas de frango, 2 meio-peitos
		\item Óleo
		\item Sal
	\end{itemize} }{
	\begin{enumerate}
		\item Cortar bem pequeno o alho e a cebola. Colocar ambos para dourar um pouco
		      de tempo. Colocar uma colher de chá de sal.
		\item Colocar o frango, sem tirar a pele ou qualquer outra coisa.
		\item Colocar a água. Cerca de 2 litros, ou até cobrir tudo. Deixar a panela
		      no fogo alto até a água começar a ferver, depois colocar no fogo médio.
		\item Tempo total é de cerca de 1h depois de começar a ferver. Mexer no frango
		      de vez em quando para ver a textura e possivelmente abrir os pedaços para
		      ajudar a cozinhar.
		      \begin{itemize}
			      \item Dá para remover o peito antes dos outros porque deve cozinhar antes, e
			            se ficar tempo demais fica duro.
		      \end{itemize}
		\item Estará pronto quando notar que a carne está saindo dos ossos.
		\item Coar tudo por uma peneira. Reservar o frango para fazer uma torta ou
		      alguma coisa assim.
	\end{enumerate} }
