\receitaemoji{Creme Brulé}{
\begin{itemize}
	\item 320g de creme de leite fresco
	\item Essência de baunilha
	\item 4 gemas
	\item 60g de açúcar fino
\end{itemize}
}{
\begin{enumerate}
	\item Esquentar o creme de leite junto com a baunilha
	\item Bater as gemas separadamente e acrescentar o açúcar
	\item Verter o creme quente sobre as gemas e ao mesmo tempo ir batendo
	      constantemente (temperagem) com um fouet. Evitar formar espuma
	      \begin{itemize}
		      \item É muito importante ir batendo, senão coagula e se perde a cremosidade.
	      \end{itemize}
	\item Verter os ramequins até preencher 2/3.
	\item Aquecer o forno a 150/180 $^\circ$C
	\item Colocar os ramequins numa forma e acrescentar água quente até no máximo
	      metade da altura dos ramequins.
	      \begin{itemize}
		      \item É basicamente um cozimento a banho maria no forno.
		      \item Se não tiver ramequins que suportam altas temperaturas, talvez dê para
		            utilizar xícaras ou potinhos de vidro.
	      \end{itemize}
	\item Levar ao forno por 30 a 40 minutos. Remover
	\item Esfriar na bancada. Depois, levar à geladeira sem cobrir, para não
	      formar uma película.
	\item Antes de servir, espalhar açúcar na superfície e passar o maçarico para
	      caramelizar.
\end{enumerate}
}
