\receitaemoji[\testado \karlAprova \karenAprova]{Bife acebolado}{
	\begin{itemize}
		\item bifes de contra-filé ou alcatra
		\item Cebola\footnote{A quantidade depende da quantidade de carne e do
			      gosto. Eu gosto de bastante, então para 1 bife pequeno, é um quarto de cebola}
		\item Sal a gosto
		\item Alho
	\end{itemize}
}{
	\begin{enumerate}
		\item Cortar a cebola em rodelas
		\item Cobrir os bifes com alho triturado e sal
		\item Colocar um fio de óleo na frigideira e deixar aquecer em fogo alto
		\item Colocar as cebolas e deixar por $\pm$ 5 minutos, virando de vez em
		      quando para não queimar.
		\item Colocar o bife na frigideira, fazendo um espaço na cebola
		\item Deixar por um bom tempo de um lado do bife, para selá-lo. Girar o bife
		      para pegar um pouco das áreas de cebola.
		\item Virar o bife quando um dos lados tiver terminado, dourado. Idealmente
		      não será virado novamente, mas não é ruim de fazer. Ir virando as cebolas
		\item Remover quando os dois lados estiverem com uma boa cor e a temperatura
		      indicar cozimento.
		\item Caso falte óleo em alguma etapa, pode colocar mais um pouco. Isso
		      pode ser feito quando o som da fritura estiver esquisito.
		\item Caso a cebola fique pronta antes, para não removê-la e ela ficar fria,
		      pode colocar sobre a carne.
	\end{enumerate}
	\fotoreceita{0.5\textwidth}{./Fotos/alcatra_acebolada}
}
