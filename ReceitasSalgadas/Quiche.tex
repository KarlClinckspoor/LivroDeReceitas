\receitatrescols{Quiche}{
	\textsc{Massa para quiche salgada \checkmark }

	\begin{itemize}
		\item 200 g de farinha
		\item 100 g de manteiga sem sal gelada
		\item 1 ovo batido
		\item 2 ou 3 colheres de água gelada
		\item sal
	\end{itemize}

	\textsc{Recheio 1: Queijo branco, parmesão e alho poró \checkmark}

	\begin{itemize}
		\item 150 g de queijo minas frescal
		\item 50 g de parmesão
		\item 50 mL de leite
		\item 50 mL de creme de leite 35\% (não usar de caixinha, pois talha em altas
		      temperaturas)
		\item 2 ovos batidos
		\item orégano
		\item 1 xícara de alho poró em fatias
	\end{itemize}
}{
	\textsc{Recheio 2: Muçarela e tomate seco}

	\begin{itemize}
		\item 150 g de muçarela ralada
		\item 2 ovos batidos
		\item 50 mL de leite
		\item 50 mL de creme de leite 35\%
		\item noz moscada
		\item manjericão fresco picado
		\item 4 metades de tomate seco
	\end{itemize}

	\textsc{Recheio 3: Quiche Lorraine}

	\begin{itemize}
		\item 250 g de bacon em tiras, fritos e secos em papel toalha
		\item 3 ovos
		\item 250 mL de creme de leite 35\%
		\item 100 g de queijo gruyere ralado
	\end{itemize}
}{
	\textsc{Recheio 4: Batata e bacon}

	\begin{itemize}
		\item 1 batata cortada em fatias finas
		\item 100 g de bacon em tiras, fritos e secos em papel toalha
		\item 2 ovos
		\item 120 mL de creme de leite 35\%
		\item 100 mL de leite
		\item 1 colher de chá de salsinha fresca picada
		\item 50 g de queijo emmental ralado
	\end{itemize}
}{

	\textsc{Preparo da massa, pré-assamento}

	\begin{enumerate}
		\item Pré-aquecer o forno a 200\grau{} C.
		\item Peneirar a farinha e uma pitada sal. Colocar no
		      processador.\footnote{Pode ser o processador parrudinho baixo}
		\item Acrescentar a manteiga em cubos ao processador no modo pulse até formar
		      uma consistência de farofa.
		\item Acrescentar o ovo batido e depois a água até formar uma bola. Colocar
		      duas colheres, e depois a terceira. Se tiver dúvida sobre colocar mais
		      água, não colocar. Vai formar uma massa firme, mas mole, não esfarelenta. Não
		      manipular muito, envolver em filme plástico e guardar na geladeira por 1
		      hora.
		\item Enquanto espera, preparar o recheio.
		\item Após 1 hora, abrir com rolo sobre papel manteiga.\footnote{Abrir o
			      início com uma garrafa/copo de vidro também serve}
		\item Virar a massa para a forma com o auxílio do papel, moldar ajustando as
		      bordas e preenchendo as falhas com pedacinhos de massa. Não precisa
		      untar. Tentar fazer os cantos ficarem mais finos para não acumular
		      massa que não vai assar direito. Levantar bastante as bordas e
		      distribuir a massa das bordas para homogeneizar a altura.
		      \begin{itemize}
			      \item Se assar assim, a massa vai descer e o fundo vai subir, então é
			            necessário assar com peso e fazer vários furinhos na base com
			            um garfo.
		      \end{itemize}
		\item Colocar feijões \textbf{sobre papel manteiga}, sobre a massa, cobrindo todo o
		      fundo e uns 2 ou 3 cm de altura.
		\item Assar por 10 minutos com os feijões, depois retirá-los, e depois assar
		      por mais 10 minutos.
		\item O recheio só pode ser colocado com a massa pré-assada.
	\end{enumerate}

	\textsc{Assamento dos recheios}

	\begin{itemize}
		\item[Recheio 1] Ralar o requeijão e esfarelar/cortar o queijo minas. Cortar o
		      alho poró, selecionando a parte mais branca, perto da base. Não colocar
		      rodelas inteiras, separá-las. Colocar o orégano. Depois, cobrir com a
		      mistura de ovos batidos, leite e creme de leite. Assar por 35/45 minutos a
		      180\grau{} C. Caso acredite que o forno é quente, assar por 25-30 e
		      checar depois de 5 em 5. O ponto final será atingido quando a massa
		      estiver dourada. O ovo irá cozinhar depois de aproximadamente 20 minutos.
		\item[Recheio 2] Distribuir na massa assada o queijo, tomate seco e
		      manjericão. Bater os ovos, leite e creme, temperar com um pouco de sal e noz
		      moscada e cobrir. Assar por 35/45 minutos a 180\grau{} C.
		\item[Recheio 3] Bater os ovos e o creme e temperar com pimenta e noz moscada.
		      Espalhar o queijo e o bacon na massa e cobrir com a mistura de ovos. Assar
		      por 35/45 minutos a 180\grau{} C.
		\item[Recheio 4] Bater os ovos, leite e o creme e temperar com salsinha.
		      Espalhar o bacon e o queijo na massa e cobrir com a mitura de ovos. Assar
		      por 35/45 minutos a 180\grau{} C.
	\end{itemize}

	\fotoreceita{0.4\textwidth}{./Fotos/quiche_poro}
}