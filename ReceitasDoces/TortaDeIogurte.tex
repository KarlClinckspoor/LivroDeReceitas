\receitaemoji{Torta de Iogurte}{
	\begin{itemize}
		\item 4 ovos separados
		\item 70 g de açúcar
		\item 350 g de iogurte grego
		\item 40 g de maizena
		\item 4 g de fermento royal
		\item raspas de limão
		\item baunilha
	\end{itemize} }{
	\begin{enumerate}
		\item Bater as claras em neve e reservar. Demora menos do que parece, na mão! Em torno de 5 minutos.
		\item Bater as gemas com o açúcar até ficar bem claro e aumentar o volume.
		\item Acrescentar o iogurte, as raspas e a baunilha às gemas batidas e misturar bem
		\item Acrescentar a maizena junto com o fermento à mistura anterior até incorporar bem.
		\item Acrescentar as claras em neve aos poucos, com auxílio de uma espátula. Não usar batedeira. Usar
		      movimentos largos, para não remover o ar.\footnote{Caso não tenha espaço no recipiente da massa para
			      fazer essa transferência, transferir a clara em neve para um prato, e a massa para o recipiente da
			      clara, depois incorporar.}
		\item Untar uma forma de 22 cm de diâmetro com manteiga.
		\item Colocar em uma forma de 22 cm de diâmetro.
		\item Assar por 40-45 minutos a 170\grau{} C. A massa não fica muito dourada. Ela fica bem elástica assim
		      que sai do forno. Usar um palito para checar se está pronto.
		\item Depois deixar esfriar até temperatura ambiente sem a tampa. Pode consumir, mas fica melhor frio. A
		      massa afunda, é natural.
	\end{enumerate} }
