\receitaemoji[\testado \karlAprova]{Lemon Curd}{
  \textsc{Conjunto A}
  \begin{itemize}
  \item 125 mL de suco de limão siciliano\footnote{Pode ser taiti, mas é mais
      ácido que o siciliano}
  \item 100 mL de creme de leite fresco
  \item 50 g de açúcar branco
  \item 2 ovos inteiros + 2 gemas
  \item raspas de limão siciliano
  \end{itemize}

  \textsc{Conjunto B}
  \begin{itemize}
  \item 125 mL de suco de limão
  \item 50 g de açúcar
  \item 85 g de manteiga sem sal
  \item 2 ovos inteiros + 2 gemas
  \item raspas de limão
  \end{itemize}
  
  \textsc{Conjunto C}
  \begin{itemize}
  \item 100 mL de suco de limão
  \item 4 ovos grandes
  \item 100 g de açúcar
  \item 100 mL de creme de leite
  \item raspas de limão
  \end{itemize}

  \textsc{Conjunto D}
  \begin{itemize}
  \item 100 mL de suco de limão
  \item 100 g de açúcar
  \item 100 mL de creme de leite fresco
  \item 4 ovos
  \item raspas de limão
  \end{itemize}
}{
  \textsc{Método micro-ondas}

  \begin{enumerate}
  \item Misturar tudo menos as rapas com um fouet e levar ao micro-ondas por 3 a
    4 minutos, removendo a cada 30 segundos e mexendo com o fouet.
  \item Se esquentar demais antes de misturar, vai formar grumos e perder a
    textura de creme
  \item Quando estiver espesso, acrescentar as rapas.
  \item Esperar esfriar e servir. Dura 4-5 dias na geladeira.
  \end{enumerate}
}