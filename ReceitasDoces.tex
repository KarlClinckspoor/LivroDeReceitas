\receita{Creme Brulé}{
	\begin{itemize}
		\item 320g de creme de leite fresco
		\item Essência de baunilha
		\item 4 gemas
		\item 60g de açúcar fino
	\end{itemize}
}{
	\begin{enumerate}
		\item Esquentar o creme de leite junto com a baunilha
		\item Bater as gemas separadamente e acrescentar o açúcar
		\item Verter o creme quente sobre as gemas e ao mesmo tempo ir batendo
		      constantemente (temperagem) com um fouet. Evitar formar espuma
		      \begin{itemize}
			      \item É muito importante ir batendo, senão coagula e se perde a cremosidade.
		      \end{itemize}
		\item Verter os ramequins até preencher 2/3.
		\item Aquecer o forno a 150/180 $^\circ$C
		\item Colocar os ramequins numa forma e acrescentar água quente até no máximo
		      metade da altura dos ramequins.
		      \begin{itemize}
			      \item É basicamente um cozimento a banho maria no forno.
			      \item Se não tiver ramequins que suportam altas temperaturas, talvez dê para
			            utilizar xícaras ou potinhos de vidro.
		      \end{itemize}
		\item Levar ao forno por 30 a 40 minutos. Remover
		\item Esfriar na bancada. Depois, levar à geladeira sem cobrir, para não
		      formar uma película.
		\item Antes de servir, espalhar açúcar na superfície e passar o maçarico para
		      caramelizar.
	\end{enumerate}
}

\receita{Overnight Oats\checkmark}{

	\begin{multicols}{2}

		\textsc{receita 1}\checkmark
		\begin{itemize}
			\item morango amassado
			\item açúcar ou maple ou mel
			\item 0,5 xícara de aveia
			\item 0,5 xícara de iogurte (grego)
			\item 0,5 limão espremido
			\item 0,5 xícara de leite
			\item Cobertura: morangos e framboesas
		\end{itemize}

		\textsc{receita 2}
		\begin{itemize}
			\item raspas de meia laranja
			\item açúcar ou maple ou mel
			\item amora preta amassada
			\item 15g de gengibre cristalizado
			\item 1 frasco de iogurte
			\item Cobertura: Amora preta
		\end{itemize}

		\textsc{receita 3}
		\begin{itemize}
			\item 10 g de chocolate em pó
			\item 20 g de aveia
			\item canela
			\item açúcar ou maple
			\item essência de baunilha
			\item 15 g de manteiga de amendoim
			\item 1 frasco de iogurte
			\item Cobertura: raspas de chocolate, framboesa e manteiga de amendoim
		\end{itemize}

		\textsc{receita 4}
		\begin{itemize}
			\item 50 g de manga amassada
			\item 1 colher sopa de coco ralado
			\item 20 g de aveia
			\item essência de baunilha
			\item açúcar ou maple
			\item 1 frasco de iogurte
			\item Cobertura: manga em cubinhos e coco ralado
		\end{itemize}

		\textsc{receita 5}
		\begin{itemize}
			\item 1 cenoura ralada
			\item 20 g de aveia
			\item canela, noz moscada, baunilha
			\item maple ou açúcar
			\item 1 frasco de iogurte
			\item Cobertura: banana e nozes
		\end{itemize}

		\textsc{receita 6}
		\begin{itemize}
			\item 40 g de abacaxi picado
			\item 1 colher de sopa de coco ralado
			\item essência de baunilha
			\item 1 frasco de iogurte
			\item 1 colher de sopa de mel
			\item 20 g de aveia
		\end{itemize}

		\textsc{receita 7}
		\begin{itemize}
			\item 10 g de chocolate em pó
			\item 1 colher de sopa de manteiga de amêndoas
			\item pitada de sal
			\item 1 frasco de iogurte
			\item 1 colher de sopa de mel
			\item 1 colher de sopa de coco ralado
			\item 20 g de aveia
		\end{itemize}

		\textsc{receita 8}
		\begin{itemize}
			\item 1 pote de iogurte
			\item Meia xícara de aveia
			\item Mel, aprox 20mL
			\item Chocolate meio amargo ralado
			\item Fruta congelada
		\end{itemize}

	\end{multicols}
} {\begin{enumerate}
		\item Misturar tudo e guardar na geladeira no dia anterior.
		\item No dia seguinte colocar a cobertura.
	\end{enumerate} }


\receita{Crepe\label{rec:crepe}}{
	\begin{itemize}
		\item 1 ovo
		\item 2 colheres de chá de açúcar
		\item 1 pitada de sal
		\item 250 mL de leite integral
		\item 100g de farinha de trigo
		\item 20g de manteiga mole, derretida
		\item essência de baunilha, licor de laranja ou qualquer aromatizante de
		      preferência
	\end{itemize}
}{
	\begin{enumerate}
		\item Juntar o ovo, açúcar, sal e bater com fouet. Mixer elétrico pode
		      desvirtuar a massa.
		\item Acrescentar 100mL de leite e misturar.
		\item Acrescentar a farinha de mexer bem até desfazer os grumos.
		\item Acrescentar aos poucos o restante do leite de misturar bem.
		\item Acrescentar a manteiga e misturar.
		\item Acrescentar o aromatizante.
		\item Deixar a massa repousar por pelo menos 1h antes de fritar. Ideal é
		      preparar com 5 horas de antecedência.
		\item Untar bem uma frigideira e aquecer em fogo médio/alto. Colocar uma
		      quantidade que preencha a frigideira bem finamente. Geralmente não precisa
		      untar a frigideira novamente.
		\item Colocar alguma cobertura se desejar. No caso do molho Suzette (receita
		      \ref{rec:molho_suzette}), se dobra o crepe em 4 e depois mergulha no molho,
		      ambos quentes, e é servido imediatamente.
	\end{enumerate}
}

\receita{Molho Suzette para Crepe\label{rec:molho_suzette}}{
	\begin{itemize}
		\item 150 mL de suco de laranja
		\item 60 g de manteiga sem sal
		\item 40 g de açúcar
		\item 1 colher de sopa de licor de laranja
		\item raspas de laranja
	\end{itemize}
}{
	\begin{enumerate}
		\item Derreter a manteiga em fogo médio e acrescentar o açúcar até
		      caramelizar.
		\item Acrescentar o suco de laranja, o licor e as raspas.
	\end{enumerate}
}

\receita{Panqueca doce}{

	\textsc{30 panquecas}

	\begin{itemize}
		\item 3 ovos
		\item Meio litro de leite integral
		\item 40g de manteiga sem sal
		\item 40g de açúcar branco
		\item 2g de sal
		\item 225g de farinha de trigo
		\item rum, licor ou essência de baunilha ou aromatizante.
	\end{itemize}

	\textsc{10 panquecas}
	\begin{itemize}
		\item 1 ovo
		\item 0,3 xícara de leite
		\item 0,3 de xícara de creme de leite
		\item 1 colher de sopa de açúcar
		\item 0,75 xícara de farinha de trigo
		\item 1 pitada de sal
		\item 1 colher de sopa de manteiga derretida
		\item 0,5 colher de chá de essência de baunilha ou aromatizante.
	\end{itemize}
}{
	\begin{enumerate}
		\item Peneirar a farinha junto com o sal e o açúcar
		\item Fazer um espaço no meio da farinha, acrescentar os ovos e ir misturando
		      com um fouet. Um misturador elétrico pode desvirtuar.
		\item Colocar o leite aos poucos para não formar pelotas, enquanto mistura
		\item Em panela separada, derreter a manteiga e juntar à massa, misturar.
		\item Acrescentar o aromatizante.
		\item Descansar a massa na geladeira por no mínimo 1h, idealmente 5.
		\item Untar uma frigideira e levar ao fogo médio/alto.
		\item Colocar uma quantidade de massa suficiente para cobrir o fundo da
		      frigideira, bem fino.
		\item Virar com uma espátula quando as bordas começarem a se soltar.
	\end{enumerate}
}

\receita{Bolo de chocolate}{ % todo: adequar o texto.
	\textsc{Bolo de chocolate}
	\begin{itemize}
		\item 4 ovos
		\item 1 xícara (chá) de açúcar
		\item 1 xícara (chá) de chocolate em pó
		\item 1 xícara (chá) de óleo
		\item 1 xícara (chá) de água
		\item 2 xícaras (chá) de farinha de trigo
		\item 1 colher (sopa) de fermento
		\item Manteiga, farinha e chocolate para untar e polvilhar
	\end{itemize}
	\textsc{Calda de ganache}
	\begin{itemize}
		\item 200g de chocolate meio amargo
		\item 3/4 de xícara (chá) de creme de leite fresco
	\end{itemize}
}{
	\begin{enumerate}
		\item Preaqueça o forno a 180ºC (temperatura média). Unte uma forma redonda ou
		      de pudim com manteiga, formando uma camada fina e uniforme.
		\item Faça uma misturinha meio a meio de chocolate em pó e farinha, e polvilhe
		      a forma toda. Desta maneira, o bolo não fica com aquela casquinha branca de
		      farinha. Reserve. Numa tigela, coloque a farinha, passando pela peneira.
		\item Na batedeira, ou numa tigela, coloque o açúcar e o chocolate em pó,
		      passando por uma peneira. Junte os ovos e o óleo. Na velocidade baixa (para
		      o chocolate não subir), bata os ingredientes, até que estejam bem
		      misturados.
		\item Aumente a velocidade e bata por mais alguns minutos. Caso prefira fazer
		      à mão, use um batedor de arame. Se estiver usando a batedeira, abaixe a
		      velocidade novamente e, aos poucos, vá adicionando a água e a farinha,
		      alternadamente, batendo apenas para misturar.
		\item Por último, adicione o fermento. Transfira a massa para a forma
		      preparada e leve ao forno preaquecido para assar por 30 minutos, até que o
		      palito saia limpo ao ser espetado no bolo.
		\item Retire do forno e deixe esfriar por 15 minutos. Coloque um prato de bolo
		      sobre a forma e, com o auxílio de um pano de prato vire de uma vez. Somente
		      quando o bolo estiver frio, espalhe a calda. Sirva a seguir.
	\end{enumerate}
}

\receita{Mousse de chocolate\checkmark}{
	\begin{itemize}
		\item 100 g de chocolate Cicao Mix (mistura de ao leite e meio-amargo)
		      \begin{itemize}
			      \item Essa quantia de chocolate foi dobrada. É possível que não precise
			            dobrar a quantidade caso use um chocolate mais duro.
		      \end{itemize}
		\item 35 g de água
		\item Gelo
	\end{itemize}
}{
	\begin{enumerate}
		\item Cortar o chocolate em pedaços grandes
		\item Misturar a água
		\item Levar ao microondas por 20 segundos.
		\item Mexer bem, levar por mais 20 segundos.
		\item Mexer, garantir que a mistura está bastante leve, sem grumos de
		      chocolate.
		\item Fazer um banho de gelo.
		\item  Colocar o chocolate no banho e bater bastante
		      com um mixer até ficar com uma consistência firme.
		\item Transferiro para um potinho
		\item Levar à geladeira por 1h para tomar consistência. Comer com moderação
	\end{enumerate}
}

\receita{Mingau de aveia\checkmark}{
	\begin{itemize}
		\item 200 mL de leite
		\item 20 g de aveia
		\item 1 punhado de cranberries
		\item (opcional) açúcar
		\item (opcional) gotas de chocolate
		\item (optional) amêndoas picadas para guarnecer
	\end{itemize}
}{
	\begin{enumerate}
		\item Misturar tudo, colocar no fogo médio, mexendo sempre, até ficar espesso
		\item Colocar o guarnecimento, se desejar.
	\end{enumerate}
}

\receita{Torta de Iogurte}{
	\begin{itemize}
		\item 4 ovos separados
		\item 70 g de açúcar
		\item 350 g de iogurte grego
		\item 40 g de maizena
		\item 4 g de fermento royal
		\item raspas de limão
		\item baunilha
	\end{itemize}
}{
	\begin{enumerate}
		\item Bater as claras em neve e reservar
		\item Bater as gemas com o açúcar até ficar bem claro e aumentar o volume.
		\item Acrescentar o iogurte, as raspas e a baunilha às gemas batidas e
		      misturar bem
		\item Acrescentar a maizena junto com o fermento à mistura anterior até
		      incorporar bem.
		\item Acrescentar as claras em neve aos poucos, com auxílio de uma espátula.
		      Não usar batedeira. Usar movimentos largos, para não remover o ar.
		\item Colocar em uma forma de 22 cm de diâmetro.
		\item Assar por 40-45 minutos a 170\grau{} C.
	\end{enumerate}
}

\receita{Bolo de cenoura\checkmark}{
  \begin{itemize}
  \item 3 cenouras grandes cortadas em rodelas
  \item 1 xícara de açúcar
  \item 0.5 xícara de óleo
  \item 3 ou 4 ovos
  \item 2 xícaras de farinha de trigo
  \item 1 colher de sopa de fermento
  \item 0.5 colher de chá de sal
  \item Manteiga para untar
  \end{itemize}
}{
  \begin{enumerate}
  \item Colocar as cenouras, açúcar, óleo e ovos no liquidificador, bater e
    reservar
  \item Colocar a farinha de trigo, fermento, sal numa cuia, misturar bem,
    adicionar os líquidos da etapa anterior, homogeneizar mas não bater muito.
  \item Untar a forma, colocar em forno pré-aquecido a 200\grau{} C por 40
    minutos.
  \end{enumerate}
}