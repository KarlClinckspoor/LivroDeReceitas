\receitaemoji[\pimenta \testado \karlAprova]{Stir Fry de Frango}{
	\begin{itemize}
		\item 1 meio-peito de frango
		\item Verduras apropriadas (cenoura, brócolis, acelga)
		\item 1 dente de alho
		\item Molho shoyu
		\item Óleo para fritar
		\item Gengibre, bem pouco.
	\end{itemize} }{
	\begin{enumerate}
		\item Cortar o frango em tirinhas
		\item Ralar 1 dente de alho inteiro.
		\item Colocar um pouco de óleo numa frigideira ou wok, e colocar o alho ralado
		      e gengibre por alguns segundos.
		\item Colocar o frango quando a panela já estiver quente. Virar bastante para
		      não deixar queimar.
		      \begin{itemize}
			      \item Se não estiver quente o suficiente, o frango irá perder água e vai
			            ficar seco.
			      \item Se tiver muito frango, fazer a fritura do mesmo em várias porções, de
			            acordo com o tamanho da frigideira e da potência do fogo.
		      \end{itemize}
		\item Quando dourado, retirar o frango e colocar armazenado em outro lugar.
		\item Fritar as verduras e legumes até ficarem cozidas.
		      \begin{itemize}
			      \item Colocar sempre os mais duros antes
			      \item Caso tenha brócolis, branquear (ferver água, deixar 2-3 minutos,
			            escorrer)
			      \item Caso tenha acelga, coloque por último
		      \end{itemize}
		\item Voltar o frango, aquecer um pouco, colocar 1 colher de sopa de shoyu,
		      mexer bem para incorporar. Vai ficar mais escuro do que aparenta no
		      início. \begin{itemize} \item Se tiver macarrão, colocar o shoyu antes,
			            incorporar, depois colocar o macarrão. \item Tomar cuidado com não deixar
			            a tampa do shoyu sair e derramar demais. \end{itemize}
	\end{enumerate}

	\fotoreceita{0.7\textwidth}{./Fotos/Stir_Fry.png} }

\receitaemoji[\pimenta \testado \karlAprova]{Bife com Curry}{
	\begin{itemize}
		\item Carne
		\item Alho
		\item Cebola
		\item Óleo/manteiga
		\item Curry em pó
	\end{itemize} }{
	\begin{enumerate}
		\item Cortar a carne em tiras
		\item Fritar cebola e alho bem picadinho, refogar com óleo ou manteiga,
		      colocar a carne e refogar uns minutos.
		      \begin{enumerate}
			      \item Como vai usar curry, pode soltar água da carne.
			      \item Caso não solte a água, abaixa o fogo e coloca a tampa e espera um
			            pouco. Caso não solte ainda, coloca meio copo de água quente.
		      \end{enumerate}

		\item Aí põe algo como uma colher de sopa de curry e termina de refogar. Serve
		      com arroz.
	\end{enumerate} }

\receitaemoji[\testado \karlAprova \karenAprova]{Arroz padrão na panela de arroz}{
	\begin{itemize}
		\item Arroz normal. Geralmente, para 2, um copo é o suficiente.
		\item Alho. Para 2, meio alho está bom.
		\item Cebola. Para 2, meia cebola pequena está bom
		\item Óleo. Suficiente para cobrir o fundo da panela com um filmezinho.
		\item Água quente
	\end{itemize}
}
{
	\begin{enumerate}
		\item (Opcional) Colocar a água para esquentar. Se fizer isso, diminui o
		      tempo de cozimento.
		\item Colocar o alho e cebola finamente picados numa panela com óleo para
		      dourar.
		      \begin{itemize}
			      \item Pode adicionar o alho depois para diminuir a
			            chance de queimar ele.
		      \end{itemize}
		\item Colocar o arroz e fritar um pouquinho. Transferir tudo para a panela de
		      arroz. Colocar a água na proporção de 2 copos de água para 1 copo de arroz
		      (medida da panela).
		\item O tempo médio de cozimento registrado aqui é de aproximadamente 13-15
		      minutos.
		\item O fundo da panela sempre fica grudado. Tentar achar uma maneira de
		      fazer isso parar de acontecer.
	\end{enumerate}
}

\receitaemoji[\testado \karlAprova \karenAprova]{Arroz arbório na panela de arroz}{\label{rec:arroz_arborio_panela}
	\begin{itemize}
		\item Arroz arbório, ou algum tipo de arroz para risoto
		\item Água, na proporção aproximada de 2x o ``volume'' de arroz.
	\end{itemize}
}{
	\begin{enumerate}
		\item Colocar o arroz arbório na panela e colocar a água. Deixar cozinhar.
		\item O resultado é um arroz que parece arroz japonês. Fica bem fofinho,
		      expandido, e contrasta bem com pratos mais fortes, como frango tailandês ou
		      outros \emph{stir-frys}.
	\end{enumerate}
}

\receitaemoji[\testado \karlAprova]{Pique Macho}{
	\begin{itemize}
		\item Batata para fritar
		\item Cebola roxa
		\item Tomate
		\item Pimentão (locoto)
		\item Salsicha
		\item Lombo
		\item Pimenta do reino
		\item Cominho
		\item Mostarda
		\item Shoyu (relativamente pouco, 2 colheres para 1kg de carne)
		\item Sal
		\item Óleo de girassol
		\item Ovos
		\item Maionese
		\item Ketchup
	\end{itemize} }{
	\begin{enumerate}
		\item Fritar as batatas
		\item Fritar a carne, sem se preocupar em selar. Parte do sabor do prato vem
		      do líquido soltado pela carne.
		\item Colocar o cominho, pimenta, mostarda, shoyu e sal com a carne ainda
		      vermelha.
		\item Colocar os ovos para cozinhar em outra panela.
		\item Deixar a carne cozinhar.
		\item Colocar a salsicha e esperar 5 minutos, desligar o fogo.
		\item Colocar a carne sobre as batatas fritas.
		\item Colocar as cebolas que estavam no vinagre para perder o sabor forte,
		      tomate, ovos cozidos, pimentões verdes, sobre o prato.
	\end{enumerate}

	\fotoreceita{0.7\textwidth}{./Fotos/pique_macho}

}

\receitaemoji[\testado \karlAprova \karenAprova]{Macarrão com manteiga, ovo e queijo}{
	\begin{itemize}
		\item Espaguete (80 g para uma pessoa)
		\item Sal
		\item 3/4 copo de leite
		\item 1 ovo
		\item 2 colheres de manteiga
		\item 2/3 copo de parmesão ralado
		\item Pimenta a gosto
	\end{itemize} }{
	\begin{enumerate}
		\item Cozinhar o macarrão numa panela. \begin{itemize}
			      \item Colocar sal na água. Deixar ferver. Colocar macarrão.
		      \end{itemize}
		\item Escorrer o macarrão ao ficar al dente.
		\item Devolver o macarrão na panela, abaixar a temperatura, colocar a mistura
		      de leite, ovo e manteiga, e mexer até recobrir o macarrão. \begin{itemize}
			      \item Quando eu fiz, deixei bastante tempo para garantir que o ovo estivesse
			            cozido. Ele engrumou, mas a Karen gostou de qualquer forma.
		      \end{itemize}
		\item Remover o calor. Colocar queijo e misturar. Se estiver pastoso demais,
		      colocar mais lente.
	\end{enumerate} }

\receitaemoji[\testado \karlAprova \karenAprova]{Caldo de frango\label{rec:caldo_frango}}{
	\begin{itemize}
		\item 3 dentes de alho
		\item 1 cebola pequena
		\item 2 coxas de frango, 2 sobrecoxas de frango, 2 meio-peitos
		\item Óleo
		\item Sal
	\end{itemize} }{
	\begin{enumerate}
		\item Cortar bem pequeno o alho e a cebola. Colocar ambos para dourar um pouco
		      de tempo. Colocar uma colher de chá de sal.
		\item Colocar o frango, sem tirar a pele ou qualquer outra coisa.
		\item Colocar a água. Cerca de 2 litros, ou até cobrir tudo. Deixar a panela
		      no fogo alto até a água começar a ferver, depois colocar no fogo médio.
		\item Tempo total é de cerca de 1h depois de começar a ferver. Mexer no frango
		      de vez em quando para ver a textura e possivelmente abrir os pedaços para
		      ajudar a cozinhar.
		      \begin{itemize}
			      \item Dá para remover o peito antes dos outros porque deve cozinhar antes, e
			            se ficar tempo demais fica duro.
		      \end{itemize}
		\item Estará pronto quando notar que a carne está saindo dos ossos.
		\item Coar tudo por uma peneira. Reservar o frango para fazer uma torta ou
		      alguma coisa assim.
	\end{enumerate} }

\receitaemoji[\testado \karlAprova \karenAprova]{Risoto de parmesão}{
	\begin{itemize}
		\item 2,5 copos de caldo de frango (vide receita \ref{rec:caldo_frango})
		\item 0,75 copos de água
		\item 2 colheres de sopa de manteiga
		\item 0,5 cebola grande, finamente cortada
		\item Sal, pimenta
		\item 0,5 dente de alho, cortado finamente
		\item 1 copo de arroz para risoto (arbório ou outro)
		\item 25-30g de parmesão ralado
		\item 1 colher de sopa de salsinha e cebolinha
		\item 0,5 colher de sopa de suco de limão
	\end{itemize} }{
	\begin{enumerate}
		\item Ferver o caldo e a água em panela grande em fogo quente. Depois reduzir
		      o calor para um fervor gentil.
		\item Derreter metade da manteiga numa panela\footnote{De preferência de
			      ferro, mas pode ser alguma com uma massa térmica maior que uma panela
			      comum} em calor médio. Colocar a cebola e 0,75 colher de sopa de sal. Ir
		      mexendo por 5 a 7 minutos.
		\item Colocar o alho até ficar cheiroso, 30 segundos.
		\item Colocar o arroz e cozinhar, mexendo lentamente até os grãos ficarem
		      translúcidos nas bordas, 3 minutos.
		      \begin{itemize}
			      \item Cuidado com deixar o arroz queimar nesta etapa! Mexer bastante.
		      \end{itemize}
		\item Colocar o caldo aquecido até cobrir o arroz. Reduzir o calor para
		      médio-baixo. Ficar mexendo até absorver. Quando secar, adicionar mais caldo
		      em parcelas bem pequenas. Sempre ficar mexendo para garantir que não vai
		      queimar. Continuar até o arroz ficar cremoso e cozido, ou acabar o caldo.
		      16-19 minutos.
		\item Colocar o parmesão e mexer. Remover do calor, cobrir, e deixar parado
		      por 5 minutos.
		\item Colocar o resto da manteiga, salsinha, cebolinha e limão.
		      \begin{itemize}
			      \item O limão não foi muito bem quisto pela Karen. Dá para colocar a
			            salsinha e cebolinha, separar em duas metades, e em uma delas colocar 0,25
			            colher de sopa do suco de limão.
		      \end{itemize}
	\end{enumerate}

	\fotoreceita{0.7\textwidth}{./Fotos/Risoto.png} }

\receitaemoji[\testado \karlAprova \karenAprova]{Conchiglioni de queijo}{
	\begin{itemize}
		\item Aproximadamente 50 Conchiglioni
		\item 1 cilindro de ricota fresca. Mais molinha é melhor (cerca de 200g?)
		\item 1 pote de creme de ricota.
		\item 100g de mussarella ralada
		\item 100g de provolone ralado
		\item 25g de parmesão ralado para recheio, um pouco para polvilhar em cima
		\item molho de tomate pronto. 1 saquinho para pouco molho, 2 para banhar bem.
		\item sal
		\item nozes picadas, aproximadamente 8
	\end{itemize} }{
	\begin{enumerate}
		\item Cozinhar os conchiglioni em água com sal até ficarem no ponto.
		\item Enquanto cozinha, misturar todos os queijos juntos, com as nozes. Provar
		      se tem sal suficiente e colocar um queijo ou outro para atingir o sabor.
		      Senão, colocar sal, mas por experiência, para meu gosto, isso não é
		      necessário.
		\item Ao terminar, coar o macarrão, deixar esfriar um pouco. Começar a aquecer
		      o forno a 200 $^\circ$C.
		\item Numa travessa grande, cobrir a base com molho de tomate, para evitar do
		      macarrão queimar durante o assamento.
		\item Preencher generosamente os conchiglioni com o recheio. Colocar na
		      travessa da maneira mais eficiente possível, porque 50 conchiglioni é
		      bastante.
		\item Jogar molho por cima deles, tentando ser homogêneo, mas não é
		      necessário. Após, salpicar parmesão a gosto.
		\item Colocar no forno por 15 minutos. Se tiver duas travessas, retirar a
		      travessa de baixo, mais perto do fogo, antes, para que não queime. O
		      interessante é derreter o recheio e aquecer tudo, pois a comida já está
		      tecnicamente pronta para ser comida.
		\item Servir e comer imediatamente. O que sobrar pode ser resfriado ou
		      congelado e posteriormente aquecido no micro ondas que não há perda de
		      sabor.
	\end{enumerate}

	\fotoreceita{0.5\textwidth}{./Fotos/conchiglioni} }

\receitaemoji[\testado \pimenta \karlAprova]{Frango tailandês com manjericão}{
	\begin{itemize}
		\item 1 meio peito de frango
		\item 1 a 2 pimentas vermelhas, a critério pessoal
		\item 2 a 4 dentes de alho
		\item 1 a 2 colheres de sopa de shoyu
		\item 1 colher de chá de açúcar cristal.
		\item 1 ramo de manjericão
		\item 1 colher de sopa de molho de ostra (opcional)
	\end{itemize} }{
	\begin{enumerate}
		\item Se for fazer arroz, já colocar para cozinhar agora que vai acabar
		      terminando aproximadamente junto com o frango. Vide receita
		      \ref{rec:arroz_arborio_panela}.
		\item Cortar o frango em pedaços \emph{bite-sized} e reservar. Quanto for
		      menor o pedaço de frango, mais rápido ele ficará cozido e menor será a
		      chance de queimar os temperos. Fatias do tamanho de um dedo fino pareceram
		      apropriadas.
		\item Macerar a pimenta e o alho juntos
		      \begin{itemize}
			      \item Se utilizar alho já picado, só precisa misturar.
		      \end{itemize}
		\item Separar as folhas do manjericão do caule e lavar.
		\item Misturar o shoyu, molho de ostra e açúcar cristal numa xícara.
		\item Começar a aquecer o wok com óleo até ficar bem quente
		\item Fritar o macerado de alho e pimenta por alguns segundos, mexendo sempre
		      \begin{itemize}
			      \item Se utilizar alho já picado, cuidado que espirra bastante! Adicionar
			            com o óleo menos quente.
		      \end{itemize}
		\item Acrescentar o frango e \emph{stir fry} por 2 minutos. Não pode estar cru nem
		      cozido demais
		      \begin{itemize}
			      \item Se achar que está pouco cozido porque os pedaços estão
			            grandes, deixar cozinhar por um pouco mais. Abaixar o fogo se
			            começar a queimar a pimenta e o alho.
		      \end{itemize}
		\item Acrescentar o molho de soja, de ostra, açúcar e mais \emph{stir fry} por 15
		      segundos.
		\item Acrescentar o manjericão, misturar, e desligar o fogo imediatamente.
		\item Servir bem quente com arroz pré-preparado e guarnecer com ovo por cima
		      \begin{enumerate}
			      \item Não achei necessário temperar o ovo, pois o prato em si já possui
			            bastante sabor.
		      \end{enumerate}

	\end{enumerate}

	\fotoreceita{0.5\textwidth}{./Fotos/frango_tailandes}
}

\receitaemoji[\testado \karlAprova \karenAprova]{Panqueca salgada}{
	\begin{itemize}
		\item 3 ovos
		\item 200 mL de leite integral
		\item 100 g de farinha de trigo
		\item 50 g de manteiga derretida
		\item sal
	\end{itemize} }{
	\begin{itemize}
		\item Peneirar a farinha.
		\item Fazer um buraco no meio da farinha e colocar os ovos. Mexer com fouet.
		\item Colocar o leite aos poucos para não formar pelotas.
		\item Derreter a manteiga numa panela separada e incorporar na massa.
		\item Descansar a massa por pelo menos 1h, idealmente 5, na geladeira.
		\item Untar uma frigideira e fritar em camadas bem finas, em fogo médio/alto.
	\end{itemize} }

\receita{Ovos diabólicos}{
	\begin{itemize}
		\item Ovos
		\item Creamcheese
		\item Pimenta vermelha
		\item Sal
		\item Mostarda
	\end{itemize} }{
	\begin{enumerate}
		\item Cozinhar os ovos por 5 minutos, colocando-os em água gelada
		      posteriormente para ajudar a soltar a casca.
		\item Cotar os ovos perpendicularmente para tirar as gemas.
		\item Misturar as gemas com creamcheese, pimenta vermelha, sal, mostarda.
		\item Quando a mistura estiver bem lisa, sem grumos, colocar num saquinho,
		      cortar a ponta e preencher os espaços da gema na clara cozida.
	\end{enumerate} }

\receitaemoji[\testado \karlAprova \karenAprova]{Nhoque a bolonhesa}{

	\textsc{Nhoque}

	\begin{itemize}
		\item 600g de batatas Asterix
		\item 1 ovo
		\item 2 xícaras de chá de parmesão ralado fino
		\item 1 colher de sopa de manteiga
		\item Pimenta síria a gosto
		\item 1 colher de sopa de farinha de trigo
		\item 1 fio de azeite e sal para a água do cozimento
	\end{itemize}

	\textsc{Molho bolonhesa} \label{rec:molho_bolonhesa}

	\begin{itemize}
		\item 1 colher de sopa de óleo ou azeite
		\item 1 cebola média picada
		\item 2 dentes de alho
		\item 1 kg de carne moída
		\item Orégano
		\item Pimenta do reino
		\item Colorau
		\item Sal
		\item 1 lata de extrato de tomate
		\item 1 sachê de molho de tomate
		\item 1 xícara de chá de água
	\end{itemize} }{

	\textsc{Nhoque}

	\begin{enumerate}
		\item Cozinhar as batatas, amassar ainda quente. (com casca?)
		\item Aguardar as batatas esfriarem. Adicionar os outros ingredientes até
		      formar uma massa tipo pão leve. Se necessário, adicionar mais farinha (o que
		      observar?)
		\item Ferver água com um fio de azeite.
		\item Fazer rolinhos e cortar a massa em tubos curtos. Colocar na água
		      fervente.
		\item Quando a massa boiar, retirar da água.
	\end{enumerate}

	\textsc{Molho bolonhesa}
	\begin{enumerate}
		\item Numa panela média, refogue a cebola e o alho no óleo ou no azeite até
		      dourar.
		\item Em seguida adicione a carne moída e tempere com orégano, pimenta do
		      reino, colorau, sal a gosto.
		\item Mexer para misturar bem e refogar por aproximadamente 10 minutos.
		\item Adicionar o extrato, molho e água.
		\item Misturar, tampar a panela e deixar cozinhar por mais 5 minutos.
		\item Desligar o fogo
	\end{enumerate}

	\fotoreceita{0.7\textwidth}{./Fotos/Nhoque} }

\receitatrescols{Vinagretes}{

	\textsc{clássico}
	\begin{itemize}
		\item 1 pitada de sal
		\item 1 pitada de pimenta (do reino ou branca)
		\item 1 colher de sopa de vinagre
		\item 3 colheres de sopa de óleo vegetal
	\end{itemize}

	\textsc{mostarda}
	\begin{itemize}
		\item 1 pitada de sal
		\item 1 pitada de pimenta (do reino ou branca)
		\item 1 colher de chá de mostarda (Dijon)
		\item 1 colher de sopa de vinagre
		\item 3 colheres de sopa de óleo vegetal
	\end{itemize}
}{
	\textsc{azeite}
	\begin{itemize}
		\item 1 pitada de sal
		\item 1 pitada de pimenta (do reino ou branca)
		\item 1 colher de sopa de vinagre
		\item 1 colher de sopa de suco de limão
		\item 4 colheres de sopa de azeite
	\end{itemize}

	\textsc{balsâmico}
	\begin{itemize}
		\item 1 pitada de sal
		\item 1 pitada de pimenta (do reino ou branca)
		\item 1 colher de sopa de vinagre balsâmico
		\item 1 colher de sopa de suco de limão
		\item 3 colheres de sopa de azeite
		\item meio dente de alho finamente picado
	\end{itemize}
}{

	\textsc{chalota\footnote{Chalota é basicamente um tipo de cebola}}
	\begin{itemize}
		\item 1 pitada de sal
		\item 1 pitada de pimenta (do reino ou branca)
		\item 1 colher de sopa de vinagre
		\item 3 colheres de sopa de óleo vegetal
		\item 1 pequena chalota cortada finamente (direção do comprimento)
	\end{itemize}

	\textsc{mostarda e mel}
	\begin{itemize}
		\item 1 pitada de sal
		\item 1 pitada de pimenta (do reino ou branca)
		\item 1 colher de chá de mostarda Dijon
		\item 1 colher de chá de mel (fluído)
		\item 1 colher de sopa de vinagre de mel
		\item 3 colheres de sopa de azeite ou óleo vegetal
		\item 1 colher de chá de chalota cortada
	\end{itemize}
}
{
	\begin{enumerate}
		\item Colocar sal e pimenta numa tigela
		\item Adicionar 1 colher de sopa de vinagre até dissolver o sal
		\item Adicionar 3 colheres de sopa de óleo, e misturar até incorporar bem o
		      vinagre e o óleo.
		\item Adicionar as ervas e outros flavorizantes.
	\end{enumerate} }

\receitaemoji[\testado \pimenta \karlAprova \karenAprova]{Moussaka}{

	\textsc{Moussaka}
	\begin{itemize}
		\item 6 batatas médias
		\item 500g de carne moída
		\item 5 tomates ou molho de tomate pronto
		\item 1 cebola
		\item azeite
		\item 30 g de manteiga
		\item uma colher de chá de canela
		\item uma colher de sopa de mel
		\item noz-moscada
		\item sal, pimenta
	\end{itemize}

	\textsc{Bechamel}
	\begin{itemize}
		\item 20 g de manteiga
		\item 3 colheres de sopa rasas de farinha
		\item 350 mL de leite
	\end{itemize}
}{
	\begin{enumerate}
		\item Cortar as cebolas e refogar em panela pequena. Colocar os tomates ou
		      molho de tomate. Colocar o azeite, canela, mel, sal, pimenta. Caso seja
		      feito com tomate, reduzir por 25 minutos em fogo médio. Senão, reduza
		      somente o quanto achar necessário.
		\item Descascar e cortar as batatas em fatias finas.
		\item Dispor as batatas no fundo de uma travessa untada com manteiga alta o
		      suficiente para caber todos os ingredientes. Colocar um pouco do molho de
		      tomate nas batatas.
		\item Levar ao forno bem forte (grelha) por 8 a 12 minutos para dourar as
		      batatas.
		\item Cozinhar a carne moída na manteiga em fogo forte com sal e pimenta.
		      Cozinhar até soltar toda a água.
		\item Adicionar o molho de tomate e reduzir, em fogo baixo.
		\item Remover a batata do forno, regular a temperatura para 200\grau C.
		\item Para fazer o bechamel, em uma panela pequena, jogar 20 g de manteiga,
		      adicionar a farinha até obter uma mistura homogênea. Incorporar o leite
		      devagar, sem parar de mexer. Juntar sal, pimenta e noz moscada.
		\item Por cima da batata, colocar uma camada de carne, outra camada de batata,
		      até acabar com tudo ou encher a travessa. Cobrir com o molho bechamel.
		\item Assar a 200 \grau C por 50 minutos a uma hora. Idealmente o bechamel irá
		      ficar dourado.
	\end{enumerate}


	\fotoreceita{0.45\textwidth}{./Fotos/Moussaka.jpg} \fotoreceita{0.45\textwidth}{./Fotos/Moussaka2.jpg}
}

\receitaemoji[\testado \pimenta \karlAprova]{Lentilha com Curry}{
	\begin{itemize}
		\item 100 g de lentilha vermelha (ideal) ou verde
		\item meia cebola média picada
		\item 1 dente de alho
		\item 1/2 cubos de Golden Curry (já contém sal) ou 1 colher e meia de Curry
		      Kitano com meia colher rasa de sopa de sal. Alternativamente:
		      \begin{itemize}
			      \item 1 colher de café de pimenta chili
			      \item 1 colher de café de cominho
			      \item 1 colher de café de coentro em sementes esmagadas
			      \item 1 colher de café de sal
		      \end{itemize}
		\item meio litro de água quente
	\end{itemize}
}{
	\begin{enumerate}
		\item Lavar a lentilha e cozinhar na água por uns 20 minutos. Checar de 5 em
		      5 minutos depois de 10 minutos de cozimento. Se for vermelha, é mais rápida.
		\item Refogar a cebola e o alho até amolecer um pouco, não precisa dourar.
		\item Se não tiver o Golden curry, acrescentar os temperos à cebola para despertar o
		      sabor e não deixar queimar. Refogar por 5 segundos.
		\item Jogar a lentilha cozida com água na cebola refogada. Misturar bem. Se
		      for usar o Golden Curry, colocar 1 tablete e esperar dissolver. Provar. Se
		      precisar de mais, colocar ou sal ou mais curry.
	\end{enumerate}

	\fotoreceita{0.8\textwidth}{./Fotos/LentilhaComCurry}
}

\receitaemoji[\testado \karlAprova \karenAprova]{Vagem refogada}{
	\begin{itemize}
		\item 1 pacotinho de vagem de mercado (chuto 20 vagens)
		\item 1/3 de cebola média
		\item Meio dente de alho ou quantidade equivalente de pasta de alho
		\item Sal a gosto
	\end{itemize}
}{
	\begin{enumerate}
		\item Cortar a cebola em pedaços relativamente pequenos, cubinhos.
		\item Lavar a vagem, tirar a ponta e o fiozinho. Esse fio vai da ponta até o
		      outro lado, passando pelo lado externo. Tirar a ponta com os dedos pode
		      ajudar a remover essa parte, mas não é necessário.
		\item Colocar a cebola na frigideira com óleo, já quente, sob temperatura
		      média, e refogar por aproximadamente 5 minutos, ou até começar a dourar um
		      pouco.
		\item Colocar o alho e fritar por poucos segundos, só até deixar o aroma sair.
		      Colocar o sal.
		\item Colocar a vagem, mexer bem no começo, depois deixar tostar um pouco.
		      Virar de vez em quando para garantir cozimento por completo. Provar depois
		      de cerca de 10 minutos e ir provando depois até ficar bom.
		\item Servir imediatamente.
	\end{enumerate}
}
\receitaemoji[\testado \karenAprova]{Macarrão de abobrinha}{
	\begin{itemize}
		\item 1 abobrinha grande
		\item Molho bolonhesa (vide \ref{rec:molho_bolonhesa})
		\item (opcional) Espaguete
		\item Cebola
		\item Alho
	\end{itemize}
}{
	\begin{enumerate}
		\item Lavar a abobrinha, passar ela no equipamento para fazer os fios
		\item Refogar a cebola até quase dourar, colocar o alho e o sal, e colocar a
		      abobrinha. Mexer um pouco.
		\item Fechar com uma tampa e deixar assim por 5 minutos.
		\item Se desejado, misturar com espaguete e servir imediatamente com o molho.
	\end{enumerate}
}

\receitaemoji[\testado \karenAprova]{Tapioca de ovo}{
	\begin{itemize}
		\item Farinha de tapioca
		\item Ovo
	\end{itemize}
}{
	\begin{enumerate}
		\item Não há muito truque para esta receita. Colocar um tanto de tapioca e um
		      tanto de ovo. Misturar. A consistência deve ficar similar à uma panqueca.
		\item Colocar no fogo, deixar tomar forma, depois colocar algum recheio, como
		      queijo. Servir quente.
	\end{enumerate}
}

\receitaemoji[\testado \karlAprova \karenAprova]{Bife acebolado}{
	\begin{itemize}
		\item 2 bifes de contra-filé
		\item 1/4 de cebola média
		\item Sal a gosto
	\end{itemize}
}{
	\begin{enumerate}
		\item Colocar o bife na frigideira com um fio de óleo vegetal
		\item Colocar a cebola logo em seguida. Remover quando a carne estiver pronta.
		      \begin{itemize}
			      \item Caso a cebola ainda não esteja boa, mas a carne sim, deixar a
			            cebola sozinha por um pouco mais de tempo.
		      \end{itemize}
	\end{enumerate}
}


\receitaemoji[\testado \karlAprova \karenAprova]{Batata frita no airfryer}{
	\begin{itemize}
		\item Batata congelada
		\item Sal a gosto
	\end{itemize}
}{
	\begin{enumerate}
		\item Colocar a batata congelada no airfryer, nas condições mostradas no
		      equipamento.
		      \begin{itemize}
			      \item  Se não tiver uma pequena quantidade de óleo na superfície da
			            batata, envolvê-la com um pouco.
		      \end{itemize}
		\item Remover, colocar num prato, e colocar sal a gosto.
	\end{enumerate}
}

\receitatrescols{Quiche}{
	\textsc{Massa para quiche salgada \checkmark }

	\begin{itemize}
		\item 200 g de farinha
		\item 100 g de manteiga sem sal gelada
		\item 1 ovo batido
		\item 2 ou 3 colheres de água gelada
		\item sal
	\end{itemize}

	\textsc{Recheio 1: Queijo branco, parmesão e alho poró \checkmark}

	\begin{itemize}
		\item 150 g de queijo minas frescal
		\item 50 g de parmesão
		\item 50 mL de leite
		\item 50 mL de creme de leite 35\% (não usar de caixinha, pois talha em altas
		      temperaturas)
		\item 2 ovos batidos
		\item orégano
		\item 1 xícara de alho poró em fatias
	\end{itemize}
}{
	\textsc{Recheio 2: Muçarela e tomate seco}

	\begin{itemize}
		\item 150 g de muçarela ralada
		\item 2 ovos batidos
		\item 50 mL de leite
		\item 50 mL de creme de leite 35\%
		\item noz moscada
		\item manjericão fresco picado
		\item 4 metades de tomate seco
	\end{itemize}

	\textsc{Recheio 3: Quiche Lorraine}

	\begin{itemize}
		\item 250 g de bacon em tiras, fritos e secos em papel toalha
		\item 3 ovos
		\item 250 mL de creme de leite 35\%
		\item 100 g de queijo gruyere ralado
	\end{itemize}
}{
	\textsc{Recheio 4: Batata e bacon}

	\begin{itemize}
		\item 1 batata cortada em fatias finas
		\item 100 g de bacon em tiras, fritos e secos em papel toalha
		\item 2 ovos
		\item 120 mL de creme de leite 35\%
		\item 100 mL de leite
		\item 1 colher de chá de salsinha fresca picada
		\item 50 g de queijo emmental ralado
	\end{itemize}
}{

	\textsc{Preparo da massa, pré-assamento}

	\begin{enumerate}
		\item Pré-aquecer o forno a 200\grau{} C.
		\item Peneirar a farinha e uma pitada sal. Colocar no
		      processador.\footnote{Pode ser o processador parrudinho baixo}
		\item Acrescentar a manteiga em cubos ao processador no modo pulse até formar
		      uma consistência de farofa.
		\item Acrescentar o ovo batido e depois a água até formar uma bola. Colocar
		      duas colheres, e depois a terceira. Se tiver dúvida sobre colocar mais
		      água, não colocar. Vai formar uma massa firme, mas mole, não esfarelenta. Não
		      manipular muito, envolver em filme plástico e guardar na geladeira por 1
		      hora.
		\item Enquanto espera, preparar o recheio.
		\item Após 1 hora, abrir com rolo sobre papel manteiga.\footnote{Abrir o
			      início com uma garrafa/copo de vidro também serve}
		\item Virar a massa para a forma com o auxílio do papel, moldar ajustando as
		      bordas e preenchendo as falhas com pedacinhos de massa. Não precisa
		      untar. Tentar fazer os cantos ficarem mais finos para não acumular
		      massa que não vai assar direito. Levantar bastante as bordas e
		      distribuir a massa das bordas para homogeneizar a altura.
		      \begin{itemize}
			      \item Se assar assim, a massa vai descer e o fundo vai subir, então é
			            necessário assar com peso e fazer vários furinhos na base com
			            um garfo.
		      \end{itemize}
		\item Colocar feijões \textbf{sobre papel manteiga}, sobre a massa, cobrindo todo o
		      fundo e uns 2 ou 3 cm de altura.
		\item Assar por 10 minutos com os feijões, depois retirá-los, e depois assar
		      por mais 10 minutos.
		\item O recheio só pode ser colocado com a massa pré-assada.
	\end{enumerate}

	\textsc{Assamento dos recheios}

	\begin{itemize}
		\item[Recheio 1] Ralar o requeijão e esfarelar/cortar o queijo minas. Cortar o
		      alho poró, selecionando a parte mais branca, perto da base. Não colocar
		      rodelas inteiras, separá-las. Colocar o orégano. Depois, cobrir com a
		      mistura de ovos batidos, leite e creme de leite. Assar por 35/45 minutos a
		      180\grau{} C. Caso acredite que o forno é quente, assar por 25-30 e
		      checar depois de 5 em 5. O ponto final será atingido quando a massa
		      estiver dourada. O ovo irá cozinhar depois de aproximadamente 20 minutos.
		\item[Recheio 2] Distribuir na massa assada o queijo, tomate seco e
		      manjericão. Bater os ovos, leite e creme, temperar com um pouco de sal e noz
		      moscada e cobrir. Assar por 35/45 minutos a 180\grau{} C.
		\item[Recheio 3] Bater os ovos e o creme e temperar com pimenta e noz moscada.
		      Espalhar o queijo e o bacon na massa e cobrir com a mistura de ovos. Assar
		      por 35/45 minutos a 180\grau{} C.
		\item[Recheio 4] Bater os ovos, leite e o creme e temperar com salsinha.
		      Espalhar o bacon e o queijo na massa e cobrir com a mitura de ovos. Assar
		      por 35/45 minutos a 180\grau{} C.
	\end{itemize}

	\fotoreceita{0.4\textwidth}{./Fotos/quiche_poro}
}

\receitaemoji[\testado \pimenta \karlAprova \karenAprova]{Kebab de frigideira}{

	\begin{itemize}
		\item 325 g de carne moída (acém, patinho). $\pm$ 1 bandeja
		\item Meia cebola média para cima
		\item Sal (superestimar, já que é bastante carne)
		\item Pimenta do reino
		\item Pimenta síria
		\item Óleo vegetal
	\end{itemize}

}{
	\begin{enumerate}
		\item Misturar a cebola ralada com a carne moída numa travessa grande.
		      \footnote{Receita inspirada pelo vídeo \texttt{https://youtu.be/mVHNKkpYsMI}.}
		      \begin{itemize}
			      \item Se tiver um blender ou um processador em mãos, colocar a carne e a
			            cebola picada lá para homogeneizar bem.
		      \end{itemize}
		\item Adicionar sal (6 giradas ida e volta), pimenta do reino (3 pitadas), 1
		      colher de sopa cheia de pimenta síria, e misturar bem
		\item Colocar 4 colheres de óleo na frigideira e cobrir tudo
		\item Espalhar a carne e abrir na frigideira. Tentar formar uma boa
		      estrutura, não deixar buracos
		\item Colocar a frigideira no fogo médio
		\item Em aproximadamente 3 minutos a água irá começar a sair
		\item Após cerca de 5-8 minutos, corte em tiras de 1-2 polegadas, com um pão
		      duro ou uma espátula. Ver se não está partindo sozinha devido a alguma
		      ebulição.
		\item Quando achar que um dos lados está dourado o suficiente, virar.
		\item Quando o outro lado terminar de cozinhar, servir imediatamente.
	\end{enumerate}

	\fotoreceita{0.5\textwidth}{./Fotos/kebab_frigideira}
}

\receitaemoji[\testado \karlAprova \karenAprova]{Massa podre para torta salgada }{
	\begin{itemize}
		\item 0.5 kg farinha de trigo
		\item 200 g de manteiga sem sal
		\item 1 ovo
		\item 0.5 xícara de água gelada
		\item 0.5 colher de sal
	\end{itemize}
}{
	\begin{enumerate}
		\item Misturar bem os ingredientes e deixar a massa coberta na geladeira por
		      30 minutos
	\end{enumerate}
	\fotoreceita{0.5\textwidth}{./Fotos/torta_frango1.jpg}
	\fotoreceita{0.5\textwidth}{./Fotos/torta_frango2.jpg}
}

\receitaemoji[\testado \karlAprova \karenAprova]{Arroz japonês\label{rec:arroz_japones}}{
	\begin{itemize}
		\item 3/4 de xícara de arroz japonês (1 medida de copo arroz), mas a receita é
		      bastante flexível, pode usar mais.
	\end{itemize}
}{
	\begin{enumerate}
		\item Colocar o arroz na panela de cozimento, enxaguar 3 vezes com água da
		      torneira para tirar a farinha de arroz.\footnote{Inspirado neste vídeo: \texttt{https://www.youtube.com/watch?v=KnBD57yN2Ow}}
		\item Cobrir o arroz com água, deixar $\pm$ 1 falange do dedo indicador de
		      água sobrando.
		\item Colocar no fogo alto e deixar sem tampa até começar a ferver.
		\item Ficar de olho quando a altura da água atingir o topo do arroz. Nesse
		      ponto, abaixar o fogo ao mínimo e cobrir com uma tampa. Deixar cozinhando
		      por 10-12 minutos (12-15 minutos se for mais arroz). Se começar a subir água
		      pela panela e cair, abrir um pouquinho para aliviar, mas não muito senão seca.
		\item Depois do período terminar, tirar do fogo e deixar por 5 minutos
		      descansando
		\item Usar uma colher de arroz para abrir o arroz, fazendo movimentos de corte
		      e levantando.
	\end{enumerate}

}

\receitaemoji[\testado \karlAprova \karenAprova]{Onigiri}{
	\begin{itemize}
		\item Arroz japonês (vide receita \ref{rec:arroz_japones})
		\item Sal
		\item (opcional) Folha de alga
	\end{itemize}

	\textsc{Temperos}
	\begin{itemize}
		\item Óleo de gergelim tostado
		\item Wasabi
		\item Shoyu
		\item Temperinho de arroz japonês
	\end{itemize}

	\textsc{Recheios}
	\begin{itemize}
		\item Bacon
		\item Ovo
	\end{itemize}
}{

	\textsc{Preparo do onigiri}
	\begin{enumerate}
		\item Deixar o arroz esfriar um pouco\footnote{Inspirado no vídeo
			      \texttt{https://www.youtube.com/watch?v=phZTQHEzU6w}}
		\item Molhar a mão na água de torneira e passar um pouco de sal grosso.
		      Evitar colocar muito.
		      \begin{itemize}
			      \item Uma opção interessante é fazer a água já salgada, para diminuir a
			            presença de grãos de sal grosso no onigiri pronto. O resultado
                  é bem menos salgado do que usar sal grosso.
		      \end{itemize}
		\item Pegar um punhado de arroz e colocar na mão, deixar plano.
		\item Fazer uma pequena depressão com os dedos e colocar o recheio.
		\item Fechar as mãos no formato do bolinho. Não pressionar de modo a não
		      aglomerar o arroz.
		\item Servir. Usar algum dos temperos para dar mais sabor, se desejado
	\end{enumerate}

	\textsc{Preparo dos recheios}
	\begin{enumerate}
		\item[Bacon] Cortar o bacon em tiras finas ou cubos pequenos. Fritar somente
		      na própria gordura. Reservar.
		      %  \item[Ovo]
		\item[Ovo] Bater um ovo até ficar bem homogêneo. Esquentar a frigideira até
		      ficar quente. Se puder usar a gordura do bacon, ajuda no sabor. Colocar o
		      ovo e espalhar bem. Continuamente mexer na panela para garantir que a
		      camada de ovo ficará com a mesma espessura. Depois de um lado ficar
		      pronto, virar, cozinhar o outro. Transferir para uma tábua e cortar em
		      tiras ou em quadrados.
          \begin{itemize}
          \item Para fazer os pegadores, cortar as bordas redondas para formar
            um quadrado de ovo.
          \item Considerar o tamanho do onigiri a ser feito para cortar. O
            pegador deve passar pela parte de baixo e ir para os dois lados.
          \item Nos onigiris feitos aqui, tiras com cerca de 3/4 de tamanho
            servem para isso. O 1/4 restante pode virar recheio, assim como as
            rebarbas.
          \end{itemize}
	\end{enumerate}

	\fotoreceita{0.5\textwidth}{./Fotos/onigiri.jpg}
	\fotoreceita{0.5\textwidth}{./Fotos/onigiri2.jpg}
	\fotoreceita{0.5\textwidth}{./Fotos/onigiri3.jpg}
}

\receitaemoji[\testado \karlAprova \karenAprova]{Chapati}{

	\begin{itemize}
		\item 200 g de farinha de trigo
		\item 0,5 colher de chá de sal
		\item 1 colher de chá de azeite
		\item 160 mL de água (não exato, depende do ponto da massa)
	\end{itemize}
}{

	\begin{enumerate}
		\item Misturar bem a farinha, sal e azeite. Colocar a água aos poucos até
		      atingir o ponto da massa. Depois disso, sovar por 15-30 minutos.
		\item Deixar descansar por 15-30 minutos, coberto.
		\item Tirar porções de $\pm$ 30 g, abrir com um rolo. Deixar tudo coberto com
		      um pano para não secar.
		\item Colocar na frigideira quente, no fogo médio/alto. 30 segundos de cada
		      lado.
		\item Para terminar o assamento, colocar diretamente sobre a chama do fogão
		      por alguns segundos e quando terminar de formar as bolhas, virar e deixar
		      até ele inchar.
	\end{enumerate}
}

\receita{Naan}{
	\textsc{4 unidades pequenas}

	\begin{itemize}
		\item 120 g de farinha de trigo
		\item 2 g de bicarbonato
		\item 2 g de sal
		\item 2 g de açúcar
		\item 80 g de iogurte grego integral
		\item 22 g de azeite
		\item 50 mL de água morna
		\item (opcional) ghee, manteiga, temperos (alho, zaatar, coentro)
	\end{itemize}
}{
	\begin{enumerate}
		\item Misturar tudo e sovar por 10 minutos.
		\item Deixar descansar numa tigela coberta com pano por 2 ou 3 horas.
		\item Dividir a massa em 4 partes e formar bolas, sempre puxando as bordas
		      para dentro
		\item Descansar por mais 10 a 15 minutos
		\item Aquecer uma frigideira aderente no fogo médio
		\item Abrir uma bola com as mãos ou um rolo
		\item Passar água levemente num dos lados da massa para ficar grudento
		\item Colocar a massa na frigideira com o lado grudento para baixo.
		\item Ao levantar bolhas, virar a frigideira, segurando sobre o fogo, alto,
		      para chamuscar.
		\item Retirar com a espátulae guarnecer com ghee/manteiga/temperos a gosto.
	\end{enumerate}
}

\receitaemoji[\testado \karlAprova \karenAprova]{Shelpek}{
	\begin{itemize}
		\item 150 g de farinha de trigo
		\item Um quarto de colher de chá de fermento químico
		\item Um quarto de colher de chá de sal
		\item Um quarto de colher de chá de açúcar
		\item 85 mL de leite
		\item Azeite para fritar
	\end{itemize}
}{
	\begin{enumerate}
		\item Peneirar a farinha, fermento, sal e açúcar
		\item Misturar com leite e sovar por 5 minutos até a massa ficar lisa
		\item Repousar por 30 minutos
		\item Fazer um cilindro com a massa e dividir em 4 partes.
		\item Abrir em círculos de 1 mm de espessura
		\item Fritar em azeite em fogo médio
	\end{enumerate}
}

\receitaemoji[\testado \karenAprova]{Batata rosti/Hashbrowns }{
	\begin{itemize}
		\item 300 g de batata
		\item Manteiga
		\item Sal
		\item Queijo
	\end{itemize}
}{
	\begin{enumerate}
		\item Descascar e ralar a batata.
		\item Espremer bastante para tirar o máximo de água possível
		\item Colocar sal (4-5 viradas do moedor) e misturar bem
		\item Untar uma frigideira pequena com manteiga e colocar a metade da batata
		      ralada. Ficar com $\pm$ 1 cm de batata é o ideal
		\item Colocar fatias de queijo generosamente
		\item Recobrir com a outra metade da batata
		\item Deixar no começo em fogo alto e depois passar para médio-alto. Depois de
		      $\pm$ 10 minutos o fundo irá começar a levantar um pouco. Olhar sempre para
		      garantir. Quando dourar, virar, usando uma tampa de panela ou prato
		\item Deixar mais 10 minutos do outro lado, ou até dourar
		\item Servir imediatamente.
	\end{enumerate}
}

\receitaemoji[\testado \karlAprova]{Espaguete a carbonara}{
	\begin{itemize}
		\item 80-90 g de macarrão espaguete n\grau 3 (outros tamanhos são ok)
		\item $\pm$ 3 tiras de bacon
		\item $\pm$ 20 g de Pecorino ou Parmesão
		\item 1 ovo inteiro
		\item Pimenta do reino
	\end{itemize}
}{
	\begin{enumerate}
		\item Colocar a água para esquentar e colocar uma pitada de sal.
		\item Picar o bacon
		\item Numa tigela, bater o ovo com o queijo com um garfo.
		\item Quando a água estiver fervendo, colocar o macarrão.
		\item Após $\pm$ 1-2 minuto, começar a fritar o bacon numa frigideira.
		\item Após os pedaços de bacon ficarem crocantes, remover o excesso da gordura
		\item Checar se o macarrão já está no ponto. Se sim, transferir diretamente da
		      panela para o bacon, com o fogo desligado.
		\item Pegar uma xícara de água fervente na mão esquerda e um fouet na mão
		      direita. Bater o ovo e o queijo a medida que se coloca um fio contínuo de
		      água. Tem que ser lentamente.
		\item Depois de ficar cremoso, colocar no macarrão.
		\item A maneira tradicional de se colocar o ovo e queijo no macarrão com bacon
		      e mexer é difícil de fazer funcionar. Se achar que o ovo está cozido de
		      menos e ligar o fogo, vai coagular.
		\item Servir imediatamente.
	\end{enumerate}
}

\receita{Frango a grega com batatas}{
	\begin{itemize}
		\item Coxa, sobrecoxa e peito de frango em quartos.
		\item batatas
		\item Suco de limão
		\item Sal
		\item Alecrim picado
		\item Orégano seco
		\item Pimenta vermelha moída
		\item Azeite
	\end{itemize}
}{
	\begin{enumerate}
		\item Marinar o frango nos temperos por 1 a 2 horas.
		\item Descascar as batatas e cortá-las em pedaços relativamente pequenos
		      (quartos para batatas pequenas)
		\item Aquecer o forno a 250 \grau C.
		\item Colocar o frango com as sobras da marinada e um pouco de caldo de frango
		      numa travessa e assar por 50 minutos.
	\end{enumerate}
}

\receita{Patties de carne moída e batata}{
	\begin{itemize}
		\item 450 g de carne moída
		\item 180 g de batata
		\item 100 g de cebola
		\item 2 dentes de alho
		\item Açafrão da terra
		\item Sal
		\item Óleo vegetal de alto ponto de fulgor
	\end{itemize}
}{
	\begin{enumerate}
		\item Descascar, ralar e espremer a batata
		\item Ralar a cebola
		\item Ralar o alho
		\item Misturar tudo, formando um hamburger, secar no papel toalha.
		\item Fritar com pouco óleo
	\end{enumerate}
}

\receita{Kachapuri}{
	\textsc{Massa}
	\begin{itemize}
		\item 1 ovo
		\item 0.5 colher de chá de açúcar
		\item 0.5 colher de chá de sal
		\item 100 mL de iogurte integral
		\item 135 + 65 g de Farinha 00
	\end{itemize}

	\textsc{Recheio}
	\begin{itemize}
		\item 70\% de mussarella
		\item 30\% de parmesão ou montanhês ou emmental
	\end{itemize}
}{
	\begin{enumerate}
		\item Preaquecer o forno a 250 \grau C.
		\item Bater o ovo, açúcar e sal.
		\item Juntar a isso o iogurte integral, e bater mais.
		\item Ir peneirando a farinha com o fermento sobre a mistura, $\pm$ 65 g, até
		      formar uma massa macia.
		\item Acrescentar mais 65 g de farinha aos poucos.
		\item Formar uma bola com a massa e deixar descansando por 20 minutos
		\item Dividir a massa em 2 bolas e abri-las formando um barquinho.
		\item Recobrir com o recheio e levar ao forno por 15 minutos.
	\end{enumerate}
}

\receita{Fake chinese food}{
	\begin{itemize}
		\item Peito de frango
		\item Pimentão vermelho e verde
		\item Cebola
		\item Abobrinha em cubos ou brócolis
	\end{itemize}

	\textsc{Molho}
	\begin{itemize}
		\item Vinagre branco
		\item 1 colher de sopa de açúcar mascavo
		\item Shoyu
		\item Ketchup
		\item Alho
		\item 1 colher de sopa de maizena
		\item Óleo de gergelim
		\item Amendoim
	\end{itemize}
}{
	\begin{enumerate}
		\item Marinar o frango em shoyu, vinagre branco e 1 colher de chá de açúcar
		      mascavo
		\item Dissolver a maizena em 0.5 copo de água.
		\item Misturar os ingredientes do molho.
		\item Fritar o frango em temperatura elevada.
		\item Fritar os legumes
		\item Acrescentar o molho e depois a maizena
		\item Servir com arroz japonês
	\end{enumerate}
}

\receitaemoji[\testado \karlAprova \karenAprova]{Carne coreana}{
	\begin{itemize}
		\item 500 g de carne (filé mignon ou contra-filé)
		\item 0.5 cebola
		\item 0.5 cenoura grande
		\item 1 alho poró
		\item 1 colher de sopa de óleo de gergelim
		\item 2 colheres de sopa de shoyu
		\item 1 colher de sopa de açúcar mascavo
		\item Pimenta do reino
		\item 2 dentes de alho espremidos\footnote{Ou aproximadamente a mesma massa
			      de alho pré-picado/moído}
	\end{itemize}
}{
	\begin{enumerate}
		\item Cortar a carne em tiras ou cubos\footnote{Pode tirar a gordura}
		\item Cortar a cebola em pétalas (transversalmente)
		\item Cortar a cenoura em fatias ou tiras
		\item Cortar a parne branca do alho poró em tiras
		\item Marinar a carne e os vegetais por 1 hora com todos os outros
		      ingredientes. Deixar bem misturado numa travessa fechada.
		\item Colocar óleo num wok e colocar em fogo alto até começar um pouco o
		      fulgor do óleo.
		\item Colocar a carne e os vegetais no wok quente e selar a carne, virando
		      várias vezes. Idealmente, não irá sair muito caldo da carne.
		\item Servir com arroz japonês.
	\end{enumerate}
}

\receita{Peito de frango com molho de mostarda e estragão}{
	\begin{itemize}
		\item 2 metades de peito de frango sem pele
		\item 1 colher de sobremesa de mostarda
		\item 0.5 xícara de creme de leite
		\item 1 colher de chá de estragão seco
		\item 1 xícara de caldo de frango
		\item 1 cebola pequena picada
		\item Azeite
	\end{itemize}
}{
	\begin{enumerate}
		\item Fritar a cebola no azeite até ficar transparente.
		\item Acrescentar o caldo de frango e ferver. Abaixar o fogo
		\item Numa frigideira separada, secar os peitos e fritar por 10 minutos (5
		      minutos de cada lado). Não fritar até o fim, só precisa formar uma crosta.
		\item Colocar os peitos no caldo quente até cozinhá-los totalmente (165 \grau
		      F).
		\item Retirar os peitos e reservá-los.
		\item Acrescentar ao molho a mostarda, creme de leite e por último o estragão.
		      Não cozinhar demais.
		\item Voltar os peitos já fatiados ao molho e servir com arroz branco.
	\end{enumerate}
}

\receita{Rosbife (Filé mignon)}{
	\begin{itemize}
		\item Uma peça de filé mignon grande
		\item Óleo para fritura
		\item Grãos de mostarda
		\item Pimenta do reino
		\item Sal grosso
		\item Tomilho
	\end{itemize}
}{
	\begin{enumerate}
		\item Pegar somente a parte central, que tem espessura regular
		\item Amarrar com barbante a cada 4-5 cm.
		\item Passar óleo em volta e fritar em fogo alto numa frigideira de ferro em
		      todos os lados.
		\item Pré-aquecer o forno a 200 \grau C.
		\item Moer a mostarda, pimenta do reino e sal grosso.
		\item Adicionar a azeite e tomilho fresco
		\item Misturar esses temperos
		\item Pincelar a mistura à peça de filé
		\item Assar por 30-40 minutos até que a temperatura interna fique entre
		      150-155 \grau F.
	\end{enumerate}
}

\receita{Peito de frango com alecrim, sálvia, alho e limão}{
	\begin{itemize}
		\item Filé de peito de frango.
		\item Sal
		\item Pimenta do reino
		\item Ghee (manteiga clarificada)
		\item Alho picado
		\item Sálvia
		\item Alecrim
		\item Limão grande (suco)
		\item Manteiga
	\end{itemize}
}{
	\begin{enumerate}
		\item Temperar os filés de frango com sal e pimenta do reino.
		\item Picar o alho, sálvia e alecrim
		\item Extrair o suco do limão
		\item Fritar ligeiramente no ghee até dourar um pouco. Reservar.
		\item Na mesma frigideira ainda com ghee, fritar o alho picado e acrescentar a
		      sálvia, alecrim bem picados.
		\item Saltear um pouco e acrescentar o suco do limão.
		\item Após aquecer o suco, acrescentar a manteiga.
		\item Após derreter a manteiga, voltar o frango até o molho ficar bem incorporado.
	\end{enumerate}
}


%%% Local Variables:
%%% mode: latex
%%% TeX-master: "main"
%%% End:
