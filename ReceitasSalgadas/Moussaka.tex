\receitaemoji[\testado \pimenta \karlAprova \karenAprova]{Moussaka}{

	\textsc{Moussaka}
	\begin{itemize}
		\item 6 batatas médias
		\item 500g de carne moída
		\item 5 tomates ou molho de tomate pronto
		\item 1 cebola
		\item azeite
		\item 30 g de manteiga
		\item uma colher de chá de canela
		\item uma colher de sopa de mel
		\item noz-moscada
		\item sal, pimenta
	\end{itemize}

	\textsc{Bechamel}
	\begin{itemize}
		\item 20 g de manteiga
		\item 3 colheres de sopa rasas de farinha
		\item 350 mL de leite
	\end{itemize}
}{
	\begin{enumerate}
		\item Cortar as cebolas e refogar em panela pequena. Colocar os tomates ou
		      molho de tomate. Colocar o azeite, canela, mel, sal, pimenta. Caso seja
		      feito com tomate, reduzir por 25 minutos em fogo médio. Senão, reduza
		      somente o quanto achar necessário.
		\item Descascar e cortar as batatas em fatias finas.
		\item Dispor as batatas no fundo de uma travessa untada com manteiga alta o
		      suficiente para caber todos os ingredientes. Colocar um pouco do molho de
		      tomate nas batatas.
		\item Levar ao forno bem forte (grelha) por 8 a 12 minutos para dourar as
		      batatas.
		\item Cozinhar a carne moída na manteiga em fogo forte com sal e pimenta.
		      Cozinhar até soltar toda a água.
		\item Adicionar o molho de tomate e reduzir, em fogo baixo.
		\item Remover a batata do forno, regular a temperatura para 200\grau C.
		\item Para fazer o bechamel, em uma panela pequena, jogar 20 g de manteiga,
		      adicionar a farinha até obter uma mistura homogênea. Incorporar o leite
		      devagar, sem parar de mexer. Juntar sal, pimenta e noz moscada.
		\item Por cima da batata, colocar uma camada de carne, outra camada de batata,
		      até acabar com tudo ou encher a travessa. Cobrir com o molho bechamel.
		\item Assar a 200 \grau C por 50 minutos a uma hora. Idealmente o bechamel irá
		      ficar dourado.
	\end{enumerate}


	\fotoreceita{0.45\textwidth}{./Fotos/Moussaka.jpg} \fotoreceita{0.45\textwidth}{./Fotos/Moussaka2.jpg}
}
