\receitaemoji[\testado \karlAprova \karenAprova \karenAprova]{Falafel}{
  \textsc{Falafel}
  \begin{itemize}
  \item 225 g de grão de bico seco
  \item 1/2 cebola (cerca 1/2 xícara)
  \item 2 dentes de alho picados
  \item 1 colher de chá de sal
  \item 1/4 colher de chá de pimenta caiena
  \item (opcional) 3/2 colher de chá de semente de coentro em pó
  \item (opcional) 1 colher de chá de cominho
  \item (opcional) 3/4 xícara de coentro
  \item (opcional) 3/4 xícara de salsinha
  \item Óleo para fritura
  \end{itemize}

  \textsc{Molho de Tahine}
  \begin{itemize}
  \item 1/3 xícara de tahine
  \item 1/3 xícara de iogurte grego
  \item 1/4 xícara de suco de limão
  \item sal
  \end{itemize}

  \textsc{Liga de farinha}
  \begin{itemize}
  \item 1/4 xícara de farinha
  \item 1/3 xícara de água
  \item 2 colheres de chá de fermento químico
  \end{itemize}
}{
  \textsc{Na noite anterior}
  \begin{enumerate}
  \item Colocar o grão de bico submerso em 2-3 polegadas de água durante a noite
  \end{enumerate}
  
  \textsc{Na data}
  \begin{enumerate}
  \item Misturar o tahine, iogurte grego, suco de limão. Colocar água para diluir se estiver muito grosso.
  \item Colocar os temperos do falafel num processador e bater até formar uma pasta.
  \item Colocar o grão de bico e bater. Não é para formar uma pasta, mas é para ficar com alguns pedaços
    relativamente grandes, como aveia.\footnote{Tomar cuidado se tiver material demais. Separar em duas
      metades e depois juntá-las, misturando bem.}
  \item Misturar a farinha e água numa tigela. Levar ao microondas e aquecer 10 segundos, remover e mexer até
    homogeneizar. Repetir isso até formar uma pasta que sustenta pequenas montanhas. Colocar o fermento e
    misturar novamente.
  \item Misturar essa liga de farinha ao falafel. Vai desaparecer.
  \item Pegar algumas superfícies e fazer bolinhas ou discos de falafel. Tentar manter os tamanhos
    consistentes. Eu tive relativamente sucesso utilizando duas colheres e rolando nas mãos.
  \item Aquecer o óleo a 320 \grau F. É importante manter a temperatura nessa faixa! Não deixar aquecer demais
    nem resfriar, por isso precisa controlar o fogo.
  \item Colocar as bolinhas de falafel com cuidado, utilizando uma colher furada. Mexer neles de vez em quando
    para impedir que grudem um no outro, nas paredes ou na base.
  \item Quando estiverem próximos de prontos, começarão a flutuar até o topo. Retirar quando o exterior
    estiver bastante dourado/marrom. Abrir e provar para garantir que estão bem cozidos.
  \item Colocar os falafels prontos sobre papel toalha.
  \end{enumerate}

  Para aquecer depois, 5 minutos a 200 \grau C no airfryer é uma boa medida
}

% https://www.youtube.com/watch?v=2l9TyNV64Zo&t=330s