\receitaemoji[\testado \pimenta \karlAprova]{Frango tailandês com manjericão}{
	\begin{itemize}
		\item 1 meio peito de frango
		\item 1 a 2 pimentas vermelhas, a critério pessoal
		\item 2 a 4 dentes de alho
		\item 1 a 2 colheres de sopa de shoyu
		\item 1 colher de chá de açúcar cristal.
		\item 1 ramo de manjericão
		\item 1 colher de sopa de molho de ostra (opcional)
	\end{itemize} }{
	\begin{enumerate}
		\item Se for fazer arroz, já colocar para cozinhar agora que vai acabar
		      terminando aproximadamente junto com o frango. Vide receita
		      \ref{rec:arroz_arborio_panela}.
		\item Cortar o frango em pedaços \emph{bite-sized} e reservar. Quanto for
		      menor o pedaço de frango, mais rápido ele ficará cozido e menor será a
		      chance de queimar os temperos. Fatias do tamanho de um dedo fino pareceram
		      apropriadas.
		\item Macerar a pimenta e o alho juntos
		      \begin{itemize}
			      \item Se utilizar alho já picado, só precisa misturar.
		      \end{itemize}
		\item Separar as folhas do manjericão do caule e lavar.
		\item Misturar o shoyu, molho de ostra e açúcar cristal numa xícara.
		\item Começar a aquecer o wok com óleo até ficar bem quente
		\item Fritar o macerado de alho e pimenta por alguns segundos, mexendo sempre
		      \begin{itemize}
			      \item Se utilizar alho já picado, cuidado que espirra bastante! Adicionar
			            com o óleo menos quente.
		      \end{itemize}
		\item Acrescentar o frango e \emph{stir fry} por 2 minutos. Não pode estar cru nem
		      cozido demais
		      \begin{itemize}
			      \item Se achar que está pouco cozido porque os pedaços estão
			            grandes, deixar cozinhar por um pouco mais. Abaixar o fogo se
			            começar a queimar a pimenta e o alho.
		      \end{itemize}
		\item Acrescentar o molho de soja, de ostra, açúcar e mais \emph{stir fry} por 15
		      segundos.
		\item Acrescentar o manjericão, misturar, e desligar o fogo imediatamente.
		\item Servir bem quente com arroz pré-preparado e guarnecer com ovo por cima
		      \begin{enumerate}
			      \item Não achei necessário temperar o ovo, pois o prato em si já possui
			            bastante sabor.
		      \end{enumerate}

	\end{enumerate}

	\fotoreceita{0.5\textwidth}{./Fotos/frango_tailandes}
}
