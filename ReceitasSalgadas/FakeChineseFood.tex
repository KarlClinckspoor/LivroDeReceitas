\receitaemoji{Fake chinese food}{
	\begin{itemize}
		\item Peito de frango
		\item Pimentão vermelho e verde, comprido. 3/4 é o suficiente.
		\item 3/4 cebola pequena, 1/2 cebola grande
		\item Abobrinha em cubos ou brócolis
	\end{itemize}

	\textsc{Molho}
	\begin{itemize}
		\item Vinagre branco. Não exagerar!
		\item 1 colher de sopa de açúcar mascavo
		\item Shoyu
		\item Ketchup
		\item 3/2 de dente de Alho
		\item 1 colher de sopa de maizena
		\item fio de óleo de gergelim
		\item 2 punhados de amendoim
	\end{itemize}
}{
	\begin{enumerate}
    \item Cortar o frango em tiras pequenas, ou cubos.
		\item Marinar o frango em shoyu, vinagre branco e 1 colher de chá de açúcar
		      mascavo. Esse molho será o precursor para o molho final.
		\item Dissolver a maizena em 0.5 copo de água.
		\item Misturar os ingredientes do molho, partindo do molho da marinação.
		\item Fritar o frango em temperatura elevada.
		\item Fritar os legumes
		\item Acrescentar o molho e depois a maizena
		\item Servir com arroz japonês
	\end{enumerate}
}
