\receitaemoji{Bolo de chocolate}{ % todo: adequar o texto.
		\textsc{Bolo de chocolate}
		\begin{itemize}
			\item 4 ovos
			\item 1 xícara (chá) de açúcar
			\item 1 xícara (chá) de chocolate em pó
			\item 1 xícara (chá) de óleo
			\item 1 xícara (chá) de água
			\item 2 xícaras (chá) de farinha de trigo
			\item 1 colher (sopa) de fermento
			\item Manteiga, farinha e chocolate para untar e polvilhar
		\end{itemize}
		\textsc{Calda de ganache}
		\begin{itemize}
			\item 200g de chocolate meio amargo
			\item 3/4 de xícara (chá) de creme de leite fresco
		\end{itemize}
	}{
		\begin{enumerate}
			\item Preaqueça o forno a 180ºC (temperatura média). Unte uma forma redonda ou
			      de pudim com manteiga, formando uma camada fina e uniforme.
			\item Faça uma misturinha meio a meio de chocolate em pó e farinha, e polvilhe
			      a forma toda. Desta maneira, o bolo não fica com aquela casquinha branca de
			      farinha. Reserve. Numa tigela, coloque a farinha, passando pela peneira.
			\item Na batedeira, ou numa tigela, coloque o açúcar e o chocolate em pó,
			      passando por uma peneira. Junte os ovos e o óleo. Na velocidade baixa (para
			      o chocolate não subir), bata os ingredientes, até que estejam bem
			      misturados.
			\item Aumente a velocidade e bata por mais alguns minutos. Caso prefira fazer
			      à mão, use um batedor de arame. Se estiver usando a batedeira, abaixe a
			      velocidade novamente e, aos poucos, vá adicionando a água e a farinha,
			      alternadamente, batendo apenas para misturar.
			\item Por último, adicione o fermento. Transfira a massa para a forma
			      preparada e leve ao forno preaquecido para assar por 30 minutos, até que o
			      palito saia limpo ao ser espetado no bolo.
			\item Retire do forno e deixe esfriar por 15 minutos. Coloque um prato de bolo
			      sobre a forma e, com o auxílio de um pano de prato vire de uma vez. Somente
			      quando o bolo estiver frio, espalhe a calda. Sirva a seguir.
		\end{enumerate}
	}
