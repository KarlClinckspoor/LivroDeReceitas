\receitaemoji[\testado \karlAprova \karenAprova]{Arroz japonês\label{rec:arroz_japones}}{
	\begin{itemize}
		\item 3/4 de xícara de arroz japonês (1 medida de copo arroz), mas a receita é
		      bastante flexível, pode usar mais.
	\end{itemize}
}{
	\begin{enumerate}
		\item Colocar o arroz na panela de cozimento, enxaguar 3 vezes com água da
		      torneira para tirar a farinha de arroz.\footnote{Inspirado neste vídeo: \texttt{https://www.youtube.com/watch?v=KnBD57yN2Ow}}
		\item Cobrir o arroz com água, deixar $\pm$ 1 falange do dedo indicador de
		      água sobrando.
		\item Colocar no fogo alto e deixar sem tampa até começar a ferver.
		\item Ficar de olho quando a altura da água atingir o topo do arroz. Nesse
		      ponto, abaixar o fogo ao mínimo e cobrir com uma tampa. Deixar cozinhando
		      por 10-12 minutos (12-15 minutos se for mais arroz). Se começar a subir água
		      pela panela e cair, abrir um pouquinho para aliviar, mas não muito senão seca.
		\item Depois do período terminar, tirar do fogo e deixar por 5 minutos
		      descansando
		\item Usar uma colher de arroz para abrir o arroz, fazendo movimentos de corte
		      e levantando.
	\end{enumerate}

  \textsc{Tempo}
  \begin{itemize}
  \item 18:08: Arroz lavado foi colocado na boca menor do fogão. 3 medidas de arroz (3 x 3/4 xícara).
  \item 18:19: Início da fervura
  \item 18:20: Nível da água atingiu o nível do arroz
  \item 18:21: Cozimento terminado
  \end{itemize}
}
